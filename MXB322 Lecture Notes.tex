%!TEX TS-program = xelatex
%!TEX options = -aux-directory=Debug -shell-escape -file-line-error -interaction=nonstopmode -halt-on-error -synctex=1 "%DOC%"
\documentclass{article}
\input{LaTeX-Submodule/template.tex}

% Additional packages & macros

% Header and footer
\newcommand{\unitName}{Partial Differential Equations}
\newcommand{\unitTime}{Semester 2, 20xx}
\newcommand{\unitCoordinator}{Dr Michael Dallaston}
\newcommand{\documentAuthors}{Tarang Janawalkar}

\fancyhead[L]{\unitName}
\fancyhead[R]{\leftmark}
\fancyfoot[C]{\thepage}

% Copyright
\usepackage[
    type={CC},
    modifier={by-nc-sa},
    version={4.0},
    imagewidth={5em},
    hyphenation={raggedright}
]{doclicense}

\date{}

\begin{document}
%
\begin{titlepage}
    \vspace*{\fill}
    \begin{center}
        \LARGE{\textbf{\unitName}} \\[0.1in]
        \normalsize{\unitTime} \\[0.2in]
        \normalsize\textit{\unitCoordinator} \\[0.2in]
        \documentAuthors
    \end{center}
    \vspace*{\fill}
    \doclicenseThis
    \thispagestyle{empty}
\end{titlepage}
\newpage
%
\tableofcontents
\newpage
%
\section{Fourier Series}
\begin{definition}[Fourier series expansion]
    The \textbf{Fourier series expansion} of \(f\) represents \(f\) by a periodic function on an interval, using trigonometric (sine and cosine) terms.

    Suppose a function \(f\left( x \right)\) is defined on an interval \(\interval{-L}{L}\), then
    the Fourier series expansion of \(f\) is given by:
    \begin{equation}\label{eq:fourier}
        f_F\left( x \right) = a_0 + \sum_{n = 1}^\infty a_n \cos{\left( \frac{n \pi x}{L} \right)} + \sum_{n = 1}^\infty b_n \sin{\left( \frac{n \pi x}{L} \right)}
    \end{equation}
    so that \(f = f_F\) on \(\interval{-L}{L}\). Here we cannot be certain about the equailty \(f = f_F\) for all \(x\) as \(f_F\) is perdioc and the convergence properties of the infinite sum are not known.
\end{definition}
Before attempting to determine the coefficients \(a_n\) and \(b_n\) for \(n \geq 1\), we must first evaluate certain useful integral relationships involving trigonometric functions.
\subsection{Integral Relationships}
\subsubsection{Sine and Cosine}
For \(n \in \N\):
\begin{align}
    \int_{-L}^L \cos{\left( \frac{n \pi x}{L} \right)} \odif{x} & = \frac{L}{n \pi} \left[ \sin{\left( \frac{n \pi x}{L} \right)} \right]_{-L}^L            \nonumber \\
                                                                & = \frac{L}{n \pi} \left[ \sin{\left( n \pi \right)} - \sin{\left( -n \pi \right)} \right] \nonumber \\
                                                                & = \frac{L}{n \pi} \left[ 0 - 0 \right]                                                    \nonumber \\
                                                                & = 0. \label{eq:cosine}
\end{align}
\begin{align}
    \int_{-L}^L \sin{\left( \frac{n \pi x}{L} \right)} \odif{x} & = -\frac{L}{n \pi} \left[ \cos{\left( \frac{n \pi x}{L} \right)} \right]_{-L}^L           \nonumber \\
                                                                & = \frac{L}{n \pi} \left[ \cos{\left( n \pi \right)} - \cos{\left( -n \pi \right)} \right] \nonumber \\
                                                                & = \frac{L}{n \pi} \left[ 1 - 1 \right]                                                    \nonumber \\
                                                                & = 0. \label{eq:sin}
\end{align}
\subsubsection{Combinations of Sine and Cosine}
% \begin{align*}
%     f\left( x \right) \cos{\left( \frac{m \pi x}{L} \right)}                      & = \begin{aligned}[t]
%                                                                                           a_0 \cos{\left( \frac{m \pi x}{L} \right)} & + \sum_{n = 1}^\infty a_n \cos{\left( \frac{n \pi x}{L} \right)} \cos{\left( \frac{m \pi x}{L} \right)} \\
%                                                                                                                                      & + \sum_{n = 1}^\infty b_n \sin{\left( \frac{n \pi x}{L} \right)} \cos{\left( \frac{m \pi x}{L} \right)}
%                                                                                       \end{aligned}                                           \\
%     \int_{-L}^L f\left( x \right) \cos{\left( \frac{m \pi x}{L} \right)} \odif{x} & = \begin{aligned}[t]
%                                                                                           a_0 \int_{-L}^L \cos{\left( \frac{m \pi x}{L} \right)} \odif{x} & + \sum_{n = 1}^\infty a_n \int_{-L}^L \cos{\left( \frac{n \pi x}{L} \right)} \cos{\left( \frac{m \pi x}{L} \right)} \odif{x} \\
%                                                                                                                                                           & + \sum_{n = 1}^\infty b_n \int_{-L}^L \sin{\left( \frac{n \pi x}{L} \right)} \cos{\left( \frac{m \pi x}{L} \right)} \odif{x}
%                                                                                       \end{aligned}
% \end{align*}
Recall the Werner formulas:
\begin{gather*}
    2 \cos{\left( \alpha \right)} \cos{\left( \beta \right)} = \cos{\left( \alpha - \beta \right)} + \cos{\left( \alpha + \beta \right)} \\
    2 \sin{\left( \alpha \right)} \sin{\left( \beta \right)} = \cos{\left( \alpha - \beta \right)} - \cos{\left( \alpha + \beta \right)} \\
    2 \sin{\left( \alpha \right)} \cos{\left( \beta \right)} = \sin{\left( \alpha - \beta \right)} + \sin{\left( \alpha + \beta \right)}
\end{gather*}
For \(n,\: m \in \N\),

\underline{Product of two cosine functions:}
\begin{equation*}
    \int_{-L}^L \cos{\left( \frac{n \pi x}{L} \right)} \cos{\left( \frac{m \pi x}{L} \right)} \odif{x} = \frac{1}{2} \int_{-L}^L \cos{\left( \frac{\left( n - m \right) \pi x}{L} \right)} \odif{x} + \frac{1}{2} \int_{-L}^L \cos{\left( \frac{\left( n + m \right) \pi x}{L} \right)} \odif{x}
\end{equation*}
As \(n + m \in \N\), the second integral term will evaluate to \(0\) due to \hyperref[eq:cosine]Equation~{\ref{eq:cosine}}.
For the first integral term, \(n - m \in \N\), except when \(n = m\) which results in \(\cos{\left( 0 \right)} = 1\).
Hence
\begin{equation*}
    \int_{-L}^L \cos{\left( \frac{n \pi x}{L} \right)} \cos{\left( \frac{m \pi x}{L} \right)} \odif{x} = \begin{cases}
        0, & n \neq m \\
        L, & n = m
    \end{cases}
\end{equation*}
\underline{Product of two sine functions:}
\begin{equation*}
    \int_{-L}^L \sin{\left( \frac{n \pi x}{L} \right)} \sin{\left( \frac{m \pi x}{L} \right)} \odif{x} = \frac{1}{2} \int_{-L}^L \cos{\left( \frac{\left( n - m \right) \pi x}{L} \right)} \odif{x} - \frac{1}{2} \int_{-L}^L \cos{\left( \frac{\left( n + m \right) \pi x}{L} \right)} \odif{x}
\end{equation*}
By the same argument,
\begin{equation*}
    \int_{-L}^L \sin{\left( \frac{n \pi x}{L} \right)} \sin{\left( \frac{m \pi x}{L} \right)} \odif{x} = \begin{cases}
        0, & n \neq m \\
        L, & n = m
    \end{cases}
\end{equation*}
\underline{Product of sine and cosine functions:}
\begin{equation*}
    \int_{-L}^L \sin{\left( \frac{n \pi x}{L} \right)} \cos{\left( \frac{m \pi x}{L} \right)} \odif{x} = \frac{1}{2} \int_{-L}^L \sin{\left( \frac{\left( n - m \right) \pi x}{L} \right)} \odif{x} + \frac{1}{2} \int_{-L}^L \sin{\left( \frac{\left( n + m \right) \pi x}{L} \right)} \odif{x}
\end{equation*}
Similary, \(n + m \in \N\) results in \(0\) for the second integral term, \(n - m \in \N\) also results in \(0\) for the first term, and when \(n = m\), as \(\sin{\left( 0 \right)} = 0\), the first term is always \(0\).
Therefore
\begin{equation*}
    \int_{-L}^L \sin{\left( \frac{n \pi x}{L} \right)} \cos{\left( \frac{m \pi x}{L} \right)} \odif{x} = 0
\end{equation*}
\subsection{Coefficients of the Fourier Series}
For \(a_0\) consider integrating \hyperref[eq:fourier]Equation~{\ref{eq:fourier}} from \(-L\) to \(L\).
\begin{align*}
    \int_{-L}^L f\left( x \right) \odif{x} & = \int_{-L}^L a_0 \odif{x} + \sum_{n = 1}^\infty a_n \int_{-L}^L \cos{\left( \frac{n \pi x}{L} \right)} \odif{x} + \sum_{n = 1}^\infty b_n \int_{-L}^L \sin{\left( \frac{n \pi x}{L} \right)} \odif{x} \\
    \int_{-L}^L f\left( x \right) \odif{x} & = 2 a_0 L                                                                                                                                                                                              \\
    a_0                                    & = \frac{1}{2L} \int_{-L}^L f\left( x \right) \odif{x}
\end{align*}
so that \(a_0\) represents the average value of \(f\) on \(\interval{-L}{L}\).

For coefficients \(a_m\), multiply the equation by \(\cos{\left( \frac{m \pi x}{L} \right)}\) before integrating.
\begin{align*}
    f\left( x \right) \cos{\left( \frac{m \pi x}{L} \right)}                      & = \begin{aligned}[t]
                                                                                          a_0 \cos{\left( \frac{m \pi x}{L} \right)} & + \sum_{n = 1}^\infty a_n \cos{\left( \frac{n \pi x}{L} \right)} \cos{\left( \frac{m \pi x}{L} \right)} \\
                                                                                                                                     & + \sum_{n = 1}^\infty b_n \sin{\left( \frac{n \pi x}{L} \right)} \cos{\left( \frac{m \pi x}{L} \right)}
                                                                                      \end{aligned}                                                                                        \\
    \int_{-L}^L f\left( x \right) \cos{\left( \frac{m \pi x}{L} \right)} \odif{x} & = \begin{aligned}[t]
                                                                                          a_0 \cancelto{0}{\int_{-L}^L \cos{\left( \frac{m \pi x}{L} \right)} \odif{x}} & + \sum_{n = 1}^\infty a_n \int_{-L}^L \cos{\left( \frac{n \pi x}{L} \right)} \cos{\left( \frac{m \pi x}{L} \right)} \odif{x}               \\
                                                                                                                                                                        & + \sum_{n = 1}^\infty b_n \cancelto{0}{\int_{-L}^L \sin{\left( \frac{n \pi x}{L} \right)} \cos{\left( \frac{m \pi x}{L} \right)} \odif{x}}
                                                                                      \end{aligned} \\
    \int_{-L}^L f\left( x \right) \cos{\left( \frac{m \pi x}{L} \right)} \odif{x} & = a_m L                                                                                                                                                                                                                     \\
    a_m                                                                           & = \frac{1}{L} \int_{-L}^L f\left( x \right) \cos{\left( \frac{m \pi x}{L} \right)} \odif{x}
\end{align*}

For coefficients \(b_m\), multiply the equation by \(\sin{\left( \frac{m \pi x}{L} \right)}\) before integrating.
\begin{align*}
    f\left( x \right) \sin{\left( \frac{m \pi x}{L} \right)}                      & = \begin{aligned}[t]
                                                                                          a_0 \sin{\left( \frac{m \pi x}{L} \right)} & + \sum_{n = 1}^\infty a_n \cos{\left( \frac{n \pi x}{L} \right)} \sin{\left( \frac{m \pi x}{L} \right)} \\
                                                                                                                                     & + \sum_{n = 1}^\infty b_n \sin{\left( \frac{n \pi x}{L} \right)} \sin{\left( \frac{m \pi x}{L} \right)}
                                                                                      \end{aligned}                                                                                                      \\
    \int_{-L}^L f\left( x \right) \sin{\left( \frac{m \pi x}{L} \right)} \odif{x} & = \begin{aligned}[t]
                                                                                          a_0 \cancelto{0}{\int_{-L}^L \sin{\left( \frac{m \pi x}{L} \right)} \odif{x}} & + \sum_{n = 1}^\infty a_n \cancelto{0}{\int_{-L}^L \cos{\left( \frac{n \pi x}{L} \right)} \sin{\left( \frac{m \pi x}{L} \right)} \odif{x}} \\
                                                                                                                                                                        & + \sum_{n = 1}^\infty b_n \int_{-L}^L \sin{\left( \frac{n \pi x}{L} \right)} \sin{\left( \frac{m \pi x}{L} \right)} \odif{x}
                                                                                      \end{aligned} \\
    \int_{-L}^L f\left( x \right) \sin{\left( \frac{m \pi x}{L} \right)} \odif{x} & = b_m L                                                                                                                                                                                                                                   \\
    b_m                                                                           & = \frac{1}{L} \int_{-L}^L f\left( x \right) \sin{\left( \frac{m \pi x}{L} \right)} \odif{x}
\end{align*}
To summarise,
\begin{gather*}
    a_0 = \frac{1}{2L} \int_{-L}^L f\left( x \right) \odif{x} \\
    a_m = \frac{1}{L} \int_{-L}^L f\left( x \right) \cos{\left( \frac{m \pi x}{L} \right)} \odif{x} \\
    b_m = \frac{1}{L} \int_{-L}^L f\left( x \right) \sin{\left( \frac{m \pi x}{L} \right)} \odif{x} 
\end{gather*}
for \(m \in \N\).
\begin{definition}[Piecewise smooth]
    A function \(f : \interval{a}{b} \to \R\), is \textbf{piecewise smooth} if each component \(f_i\) of \(f\) has a \underline{bounded derivative} \(f_i'\) which is \underline{continuous everywhere} in \(\interval{a}{b}\), except at
    a finite number of points at which left- and right-sided derivatives exist.
\end{definition}
\begin{theorem}[Convergence of piecewise smooth functions]
    If \(f\) is a periodic piecewise smooth function on \(\interval{-L}{L}\), \(f_F\) will converge to
    \begin{equation*}
        f_F\left( x \right) = \lim_{\epsilon \to 0^{+}} \frac{f\left( x + \epsilon \right) + f\left( x - \epsilon \right)}{2}
    \end{equation*}
    that is, \(f = f_F\), except at discontinuities, where \(f_F\) is equal to the point halfway between the left- and right-hand limits.
\end{theorem}
\begin{corollary}[Dirichlet conditions]
    The Dirichlet conditions provide sufficient conditions for a real-valued function \(f\) to be
    equal to its Fourier series \(f_F\) on \(\interval{-L}{L}\), at each point where \(f\) is continuous.

    The conditions are:
    \begin{enumerate}
        \item \(f\) has a finite number of maxima and minima over \(\interval{-L}{L}\).
        \item \(f\) has a finite number of discontinuities, in each of which the derivative \(f'\) exists
        and does not change sign.
        \item \(\int_{-L}^L \abs*{f\left( x \right)} \odif{x}\) exists.
    \end{enumerate}
\end{corollary}
\begin{definition}[Gibbs phenomenon]
    If \(f_F\) does not converge to \(f\) at discontiniuties \(x_i\), then the \(f_F\) converges 
    non-uniformally. For Fourier series this property is known as the \textit{Gibbs phenomenon}.
\end{definition}
\begin{note}
    When \(f\) is non-periodic, \(f_F\) converges to the periodic extension of \(f\).
    The endpoints may converge non-uniformally, corresponding to jump disontinuities in the periodic extension of \(f\).
\end{note}
\subsection{Sine and Cosine Series}
    \begin{definition}[Odd function]
        \(f\) is an \textit{odd} function if it satisfies
        \begin{equation*}
            f\left( -x \right) = -f\left( x \right)
        \end{equation*}     
    \end{definition}
    \begin{definition}[Even function]
        \(f\) is an \textit{even} function if it satisfies
        \begin{equation*}
            f\left( -x \right) = f\left( x \right)
        \end{equation*}     
    \end{definition}
    If \(f\) is an odd function on \(\interval{-L}{L}\), then the coefficients corresponding to the cosine terms will be zero.
    The Fourier series simplifies to
    \begin{equation*}
        f_F = \sum_{n = 1}^\infty b_n \sin{\left( \frac{n \pi x}{L} \right)}
    \end{equation*}
    where \(b_n = \frac{2}{L} \int_0^L f\left( x \right) \sin{\left( \frac{n \pi x}{L} \right)} \odif{x}\).

    Likewise if \(f\) is an even function on \(\interval{-L}{L}\), then the coefficients corresponding to the sine terms will be zero.
    The Fourier series simplifies to
    \begin{equation*}
        f_F = a_0 + \sum_{n = 1}^\infty a_n \cos{\left( \frac{n \pi x}{L} \right)}
    \end{equation*}
    where \(a_0 = \frac{1}{L} \int_0^L f\left( x \right) \odif{x}\) and \(a_n = \frac{2}{L} \int_0^L f\left( x \right) \sin{\left( \frac{n \pi x}{L} \right)} \odif{x}\).

    These special cases are known as the sine and cosine series expansions respectively, resulting in the \textbf{odd} or \textbf{even} periodic extension of \(f\).
\end{document}
