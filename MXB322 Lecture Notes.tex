%!TEX TS-program = xelatex
%!TEX options = -aux-directory=Debug -shell-escape -file-line-error -interaction=nonstopmode -halt-on-error -synctex=1 "%DOC%"
\documentclass{article}
\input{LaTeX-Submodule/template.tex}

% Additional packages & macros
\usepackage{calc}
\DeclareMathOperator{\Arg}{Arg}
\DeclareMathOperator{\Log}{Log}

% Header and footer
\newcommand{\unitName}{Partial Differential Equations}
\newcommand{\unitTime}{Semester 1, 2023}
\newcommand{\unitCoordinator}{Dr Michael Dallaston}
\newcommand{\documentAuthors}{Tarang Janawalkar}

\fancyhead[L]{\unitName}
\fancyhead[R]{\leftmark}
\fancyfoot[C]{\thepage}

% Copyright
\usepackage[
    type={CC},
    modifier={by-nc-sa},
    version={4.0},
    imagewidth={5em},
    hyphenation={raggedright}
]{doclicense}

\date{}

\begin{document}
%
\begin{titlepage}
    \vspace*{\fill}
    \begin{center}
        \LARGE{\textbf{\unitName}} \\[0.1in]
        \normalsize{\unitTime} \\[0.2in]
        \normalsize\textit{\unitCoordinator} \\[0.2in]
        \documentAuthors
    \end{center}
    \vspace*{\fill}
    \doclicenseThis
    \thispagestyle{empty}
\end{titlepage}
\newpage
%
\tableofcontents
\newpage
%
\section{Fourier Series}
\begin{definition}[Fourier series expansion]
    The \textbf{Fourier series expansion} of \(f\) represents \(f\) by a periodic function using trigonometric (sine and cosine) terms.
    Suppose a function \(f\left( x \right)\) is defined on an interval
    \(\interval{-L}{L}\), then the Fourier series expansion of \(f\) is
    given by:
    \begin{equation}
        \label{eq:fourier}
        f_F\left( x \right) = a_0 + \sum_{n = 1}^\infty a_n \cos{\left( \frac{n \pi x}{L} \right)} + \sum_{n = 1}^\infty b_n \sin{\left( \frac{n \pi x}{L} \right)}
    \end{equation}
    so that \(f = f_F\) on \(\interval{-L}{L}\). Note that \(f = f_F\) may not hold for all \(x\) as \(f_F\) is periodic and the convergence of the series is not guaranteed.
\end{definition}
To determine the coefficients \(a_n\) and \(b_n\), let us look at some useful integral properties.
\subsection{Integral Relationships}
\subsubsection{Sine and Cosine}
For \(n \in \Z\):
\begin{align*}
    \int_{-L}^L \cos{\left( \frac{n \pi x}{L} \right)} \odif{x} & = \frac{L}{n \pi} \left[ \sin{\left( \frac{n \pi x}{L} \right)} \right]_{-L}^L            \\
                                                                & = \frac{L}{n \pi} \left[ \sin{\left( n \pi \right)} - \sin{\left( -n \pi \right)} \right] \\
                                                                & = \frac{L}{n \pi} \left[ 0 - 0 \right]                                                    \\
                                                                & = 0.
\end{align*}
\begin{align*}
    \int_{-L}^L \sin{\left( \frac{n \pi x}{L} \right)} \odif{x} & = -\frac{L}{n \pi} \left[ \cos{\left( \frac{n \pi x}{L} \right)} \right]_{-L}^L           \\
                                                                & = \frac{L}{n \pi} \left[ \cos{\left( n \pi \right)} - \cos{\left( -n \pi \right)} \right] \\
                                                                & = \frac{L}{n \pi} \left[ 1 - 1 \right]                                                    \\
                                                                & = 0.
\end{align*}
\subsubsection{Combinations of Sine and Cosine}
Recall the Werner formulas:
\begin{gather*}
    2 \cos{\left( \alpha \right)} \cos{\left( \beta \right)} = \cos{\left( \alpha - \beta \right)} + \cos{\left( \alpha + \beta \right)} \\
    2 \sin{\left( \alpha \right)} \sin{\left( \beta \right)} = \cos{\left( \alpha - \beta \right)} - \cos{\left( \alpha + \beta \right)} \\
    2 \sin{\left( \alpha \right)} \cos{\left( \beta \right)} = \sin{\left( \alpha - \beta \right)} + \sin{\left( \alpha + \beta \right)}
\end{gather*}
For \(n,\: m \in \N\),

\underline{Product of two cosine functions:}
\begin{equation*}
    \int_{-L}^L \cos{\left( \frac{n \pi x}{L} \right)} \cos{\left( \frac{m \pi x}{L} \right)} \odif{x} = \frac{1}{2} \int_{-L}^L \cos{\left( \frac{\left( n - m \right) \pi x}{L} \right)} \odif{x} + \frac{1}{2} \int_{-L}^L \cos{\left( \frac{\left( n + m \right) \pi x}{L} \right)} \odif{x}
\end{equation*}
When:
\begin{itemize}
    \item \(n = m\): \(n - m = 0\) and \(\left( n + m \right) \in \Z\), so that the second
          term is \(0\), and the first term is \(L\).
    \item \(n \neq m\): \(\left( n - m \right),\: \left( n + m \right) \in \Z\) so that both terms evaluate to \(0\).
\end{itemize}
Therefore,
\begin{equation*}
    \int_{-L}^L \cos{\left( \frac{n \pi x}{L} \right)} \cos{\left( \frac{m \pi x}{L} \right)} \odif{x} =
    \begin{cases}
        0, & n \neq m \\
        L, & n = m
    \end{cases}
\end{equation*}
\underline{Product of two sine functions:}
\begin{equation*}
    \int_{-L}^L \sin{\left( \frac{n \pi x}{L} \right)} \sin{\left( \frac{m \pi x}{L} \right)} \odif{x} = \frac{1}{2} \int_{-L}^L \cos{\left( \frac{\left( n - m \right) \pi x}{L} \right)} \odif{x} - \frac{1}{2} \int_{-L}^L \cos{\left( \frac{\left( n + m \right) \pi x}{L} \right)} \odif{x}
\end{equation*}
By the same argument,
\begin{equation*}
    \int_{-L}^L \sin{\left( \frac{n \pi x}{L} \right)} \sin{\left( \frac{m \pi x}{L} \right)} \odif{x} =
    \begin{cases}
        0, & n \neq m \\
        L, & n = m
    \end{cases}
\end{equation*}
\underline{Product of sine and cosine functions:}
\begin{equation*}
    \int_{-L}^L \sin{\left( \frac{n \pi x}{L} \right)} \cos{\left( \frac{m \pi x}{L} \right)} \odif{x} = \frac{1}{2} \int_{-L}^L \sin{\left( \frac{\left( n - m \right) \pi x}{L} \right)} \odif{x} + \frac{1}{2} \int_{-L}^L \sin{\left( \frac{\left( n + m \right) \pi x}{L} \right)} \odif{x}
\end{equation*}
When:
\begin{itemize}
    \item \(n = m\): \(n - m = 0\) and \(\left( n + m \right) \in \Z\), so that the integral reduces to \(0\).
    \item \(n \neq m\): \(\left( n - m \right),\: \left( n + m \right) \in \Z\) so that both terms evaluate to \(0\) when integrated separately.
\end{itemize}
Therefore,
\begin{equation*}
    \int_{-L}^L \sin{\left( \frac{n \pi x}{L} \right)} \cos{\left( \frac{m \pi x}{L} \right)} \odif{x} = 0
\end{equation*}
In summary:
\begin{align}
    \int_{-L}^L \cos{\left( \frac{n \pi x}{L} \right)} \odif{x}                                        & = 0 \label{eq:cosine} \\
    \int_{-L}^L \sin{\left( \frac{n \pi x}{L} \right)} \odif{x}                                        & = 0 \label{eq:sine}   \\
    \int_{-L}^L \cos{\left( \frac{n \pi x}{L} \right)} \cos{\left( \frac{m \pi x}{L} \right)} \odif{x} & =
    \begin{cases}
        0, & n \neq m \\
        L, & n = m
    \end{cases}
    \\
    \int_{-L}^L \sin{\left( \frac{n \pi x}{L} \right)} \sin{\left( \frac{m \pi x}{L} \right)} \odif{x} & =
    \begin{cases}
        0, & n \neq m \\
        L, & n = m
    \end{cases}
    \\
    \int_{-L}^L \sin{\left( \frac{n \pi x}{L} \right)} \cos{\left( \frac{m \pi x}{L} \right)} \odif{x} & = 0
\end{align}
\subsection{Coefficients of the Fourier Series}
\subsubsection{For \texorpdfstring{\(a_0\)}{a0}}
For \(a_0\) consider integrating \hyperref[eq:fourier]{Equation~\ref{eq:fourier}} from \(-L\) to \(L\).
\begin{align*}
    \int_{-L}^L f\left( x \right) \odif{x} & = \int_{-L}^L a_0 \odif{x} + \sum_{n = 1}^\infty a_n \int_{-L}^L \cos{\left( \frac{n \pi x}{L} \right)} \odif{x} + \sum_{n = 1}^\infty b_n \int_{-L}^L \sin{\left( \frac{n \pi x}{L} \right)} \odif{x} \\
    \int_{-L}^L f\left( x \right) \odif{x} & = 2 a_0 L                                                                                                                                                                                              \\
    a_0                                    & = \frac{1}{2L} \int_{-L}^L f\left( x \right) \odif{x}
\end{align*}
so that \(a_0\) represents the average value of \(f\) on \(\interval{-L}{L}\).
\subsubsection{For \texorpdfstring{\(a_n\)}{an}}
For coefficients \(a_m\), multiply the equation by \(\cos{\left( \frac{m \pi x}{L} \right)}\) before integrating.
\begin{align*}
    f\left( x \right) \cos{\left( \frac{m \pi x}{L} \right)}                      & =
    \begin{aligned}[t]
        a_0 \cos{\left( \frac{m \pi x}{L} \right)} & + \sum_{n = 1}^\infty a_n \cos{\left( \frac{n \pi x}{L} \right)} \cos{\left( \frac{m \pi x}{L} \right)} \\
                                                   & + \sum_{n = 1}^\infty b_n \sin{\left( \frac{n \pi x}{L} \right)} \cos{\left( \frac{m \pi x}{L} \right)}
    \end{aligned}
    \\
    \int_{-L}^L f\left( x \right) \cos{\left( \frac{m \pi x}{L} \right)} \odif{x} & =
    \begin{aligned}[t]
        a_0 \cancelto{0}{\int_{-L}^L \cos{\left( \frac{m \pi x}{L} \right)} \odif{x}} & + \sum_{n = 1}^\infty a_n \int_{-L}^L \cos{\left( \frac{n \pi x}{L} \right)} \cos{\left( \frac{m \pi x}{L} \right)} \odif{x}               \\
                                                                                      & + \sum_{n = 1}^\infty b_n \cancelto{0}{\int_{-L}^L \sin{\left( \frac{n \pi x}{L} \right)} \cos{\left( \frac{m \pi x}{L} \right)} \odif{x}}
    \end{aligned}
    \\
    \int_{-L}^L f\left( x \right) \cos{\left( \frac{m \pi x}{L} \right)} \odif{x} & = a_m L                                                                                     \\
    a_m                                                                           & = \frac{1}{L} \int_{-L}^L f\left( x \right) \cos{\left( \frac{m \pi x}{L} \right)} \odif{x}
\end{align*}
\subsubsection{For \texorpdfstring{\(b_n\)}{bn}}
For coefficients \(b_m\), multiply the equation by \(\sin{\left( \frac{m \pi x}{L} \right)}\) before integrating.
\begin{align*}
    f\left( x \right) \sin{\left( \frac{m \pi x}{L} \right)}                      & =
    \begin{aligned}[t]
        a_0 \sin{\left( \frac{m \pi x}{L} \right)} & + \sum_{n = 1}^\infty a_n \cos{\left( \frac{n \pi x}{L} \right)} \sin{\left( \frac{m \pi x}{L} \right)} \\
                                                   & + \sum_{n = 1}^\infty b_n \sin{\left( \frac{n \pi x}{L} \right)} \sin{\left( \frac{m \pi x}{L} \right)}
    \end{aligned}
    \\
    \int_{-L}^L f\left( x \right) \sin{\left( \frac{m \pi x}{L} \right)} \odif{x} & =
    \begin{aligned}[t]
        a_0 \cancelto{0}{\int_{-L}^L \sin{\left( \frac{m \pi x}{L} \right)} \odif{x}} & + \sum_{n = 1}^\infty a_n \cancelto{0}{\int_{-L}^L \cos{\left( \frac{n \pi x}{L} \right)} \sin{\left( \frac{m \pi x}{L} \right)} \odif{x}} \\
                                                                                      & + \sum_{n = 1}^\infty b_n \int_{-L}^L \sin{\left( \frac{n \pi x}{L} \right)} \sin{\left( \frac{m \pi x}{L} \right)} \odif{x}
    \end{aligned}
    \\
    \int_{-L}^L f\left( x \right) \sin{\left( \frac{m \pi x}{L} \right)} \odif{x} & = b_m L                                                                                     \\
    b_m                                                                           & = \frac{1}{L} \int_{-L}^L f\left( x \right) \sin{\left( \frac{m \pi x}{L} \right)} \odif{x}
\end{align*}
To summarise,
\begin{align*}
    a_0 & = \frac{1}{2L} \int_{-L}^L f\left( x \right) \odif{x}                                       \\
    a_n & = \frac{1}{L} \int_{-L}^L f\left( x \right) \cos{\left( \frac{n \pi x}{L} \right)} \odif{x} \\
    b_n & = \frac{1}{L} \int_{-L}^L f\left( x \right) \sin{\left( \frac{n \pi x}{L} \right)} \odif{x}
\end{align*}
for \(n \in \N\).
\begin{definition}[Piecewise smooth]
    A function \(f : \interval{a}{b} \to \R\), is \textbf{piecewise smooth} if each component \(f_i\) of \(f\) has a \underline{bounded derivative} \(f_i'\) which is \underline{continuous everywhere} in \(\interval{a}{b}\), except at
    a finite number of points at which left- and right-sided derivatives exist.
\end{definition}
\begin{theorem}[Convergence of piecewise smooth functions]
    If \(f\) is a periodic piecewise smooth function on \(\interval{-L}{L}\), \(f_F\) will converge to
    \begin{equation*}
        f_F\left( x \right) = \lim_{\epsilon \to 0^{+}} \frac{f\left( x + \epsilon \right) + f\left( x - \epsilon \right)}{2}
    \end{equation*}
    that is, \(f = f_F\), except at discontinuities, where \(f_F\) is equal to the point halfway between the left- and right-hand limits.
\end{theorem}
\begin{corollary}[Dirichlet conditions]
    The Dirichlet conditions provide sufficient conditions for a real-valued function \(f\) to be
    equal to its Fourier series \(f_F\) on \(\interval{-L}{L}\), at each point where \(f\) is continuous.
    The conditions are:
    \begin{enumerate}
        \item \(f\) has a finite number of maxima and minima over \(\interval{-L}{L}\).
        \item \(f\) has a finite number of discontinuities, in each of which the derivative \(f'\) exists and does not change sign.
        \item \(\int_{-L}^L \abs*{f\left( x \right)} \odif{x}\) exists.
    \end{enumerate}
\end{corollary}
\begin{definition}[Gibbs phenomenon]
    If \(f_F\) does not converge to \(f\) at discontinuities \(x_i\), then the \(f_F\) converges
    non-uniformly. For Fourier series expansions, this property is known as the \textit{Gibbs phenomenon}.
\end{definition}
\begin{note}
    When \(f\) is non-periodic, \(f_F\) converges to the periodic extension of \(f\).
    The endpoints may converge non-uniformly, corresponding to jump discontinuities in the periodic extension of \(f\).
\end{note}
\subsection{Sine and Cosine Series}
\begin{definition}[Odd function]
    \(f\) is an \textit{odd} function if it satisfies
    \begin{equation*}
        f\left( -x \right) = -f\left( x \right)
    \end{equation*}
\end{definition}
\begin{definition}[Even function]
    \(f\) is an \textit{even} function if it satisfies
    \begin{equation*}
        f\left( -x \right) = f\left( x \right)
    \end{equation*}
\end{definition}
If \(f\) is an odd function on \(\interval{-L}{L}\), then the coefficients corresponding to the cosine terms will be zero.
The Fourier series simplifies to
\begin{equation*}
    f_F = \sum_{n = 1}^\infty b_n \sin{\left( \frac{n \pi x}{L} \right)}
\end{equation*}
where \(b_n = \frac{2}{L} \int_0^L f\left( x \right) \sin{\left( \frac{n \pi x}{L} \right)} \odif{x}\).
Likewise, if \(f\) is an even function on \(\interval{-L}{L}\), then the coefficients corresponding to the sine terms will be zero.
The Fourier series simplifies to
\begin{equation*}
    f_F = a_0 + \sum_{n = 1}^\infty a_n \cos{\left( \frac{n \pi x}{L} \right)}
\end{equation*}
where \(a_0 = \frac{1}{L} \int_0^L f\left( x \right) \odif{x}\) and \(a_n = \frac{2}{L} \int_0^L f\left( x \right) \cos{\left( \frac{n \pi x}{L} \right)} \odif{x}\).
These special cases are known as the sine and cosine series expansions respectively, resulting in the \textbf{odd} or \textbf{even} periodic extension of \(f\).
\section{Partial Differential Equations}
A partial differential equation (PDE) is a differential equation that
must be solved for an unknown function of at least two independent
variables, where the equation contains partial derivatives of the
unknown function. PDEs are characterised by several properties:
\begin{itemize}
    \item The \textbf{order} of the PDE is the order of the highest
          derivative in the equation. Furthermore, each independent
          variable can be described by its order.
    \item A PDE is \textbf{linear} if it is linear in its unknown
          function and its derivatives.
          \begin{itemize}
              \item A linear PDE has \textbf{constant} coefficients if
                    the coefficients of the linear terms do not depend
                    on the independent variables, and has
                    \textbf{variable} coefficients otherwise.
              \item A linear PDE is \textbf{homogeneous} if all terms
                    depend on the unknown function, and
                    \textbf{nonhomogeneous} otherwise.
          \end{itemize}
\end{itemize}
\subsection{Initial Boundary Value Problems}
As with ODEs, we can find the general solution to a PDE and then use
initial/boundary conditions to solve for arbitrary constants. The
number of conditions for each independent variable depends on the order
of that variable in the PDE\@. Problems with initial and boundary
conditions are called \textbf{initial boundary value problems} and are
often referred to as \textbf{IBVPs}.
\subsubsection{Boundary Condition Classification}
Boundary conditions may depend on \(u\), the gradient \(\pdv{u}{x}\),
or both, depending on the situation being modelled. The following is a
list of the different types of boundary conditions:
\begin{description}
    \item[Dirichlet] \(u\left( a,\: t \right) = C\)
    \item[Neumann] \(\pdv{u}{x} \left( a,\: t \right) = C\)
    \item[Robin] \(A u\left( a,\: t \right) + B \pdv{u}{x} \left( a,\:
          t \right) = C\)
\end{description}
where in each classification, the boundary condition is homogeneous iff \(C = 0\).
\subsection{Linear Operators and Superposition}
By linearity, we can write a PDE in terms of linear operators. For
example, we can write the PDE
\begin{equation*}
    \pdv[order=2]{u}{x} = \pdv{u}{x} + \pdv{u}{t}
\end{equation*}
as
\begin{equation*}
    L\left( u \right) = \pdv[order=2]{u}{x} - \pdv{u}{x} - \pdv{u}{t} \iff L = \pdv[order=2]{}{x} - \pdv{}{x} - \pdv{}{t}.
\end{equation*}
Similarly, we can describe initial/boundary conditions as linear or homogeneous.
\begin{theorem}[Superposition]
    If \(u_n\), \(n = 1,\: \dots,\: N\) are solutions to the homogeneous PDE \(L\left( u \right) = 0\), then any linear combination
    of these solutions is a solution to the PDE
    \begin{equation*}
        u = \sum_{n = 1}^N c_n u_n
    \end{equation*}
    where \(c_n\) are constants.
\end{theorem}
\subsection{Heat Equation}
Consider the temperature \(u\left( x,\: t \right)\) of a 1d metal rod
of length \(L\) with an initial temperature \(u\left( x,\: 0 \right) =
f\left( x \right)\) and boundary conditions \(u\left( 0,\: t \right) =
T_1\) and \(u\left( L,\: t \right) = T_2\). If we consider a small
section \(\interval{x_1}{x_2} \in \interval{0}{L}\), then the rate of
change of heat \(H\left( x,\: t \right)\) in this section is given by
\begin{align*}
    \text{Rate of change of heat energy} & = \text{Flow in} - \text{Flow out}                    \\
    \int_{x_1}^{x_2} \pdv{H}{t} \odif{x} & = Q\left( x_1,\: t \right) - Q\left( x_2,\: t \right)
\end{align*}
where \(Q\left( x,\: t \right)\) is the heat flux at time \(t\).
By making the following assumptions, we can formulate a relationship for the temperature in the rod at position \(x\) at time \(t\).
\begin{enumerate}
    \item No energy is lost in the rod.
    \item The change in heat energy is proportional to the change in
          temperature (i.e., no phase changes are present) so that the
          specific heat equation applies.
          \begin{equation*}
              \adif{H} = \rho c \adif{u} \iff \pdv{H}{t} = \rho c \pdv{u}{t}
          \end{equation*}
          where \(\rho\) is the density of the rod and \(c\) is the specific heat of the rod.
    \item The material of the rod is homogeneous, and Fourier's law of
          conduction applies.
          \begin{equation*}
              \symbfit{Q} = -\kappa \nabla \symbfit{u} \implies Q = -\kappa \pdv{u}{x}
          \end{equation*}
          where \(Q = Q\left( x,\: t \right)\) is the heat flux at time \(t\), and \(\kappa\) is the thermal conductivity of the rod.
\end{enumerate}
Using these assumptions, we find
\begin{align*}
    \int_{x_1}^{x_2} \pdv{H}{t} \odif{x}        & = Q\left( x_1,\: t \right) - Q\left( x_2,\: t \right)                               \\
    \int_{x_1}^{x_2} \rho c \pdv{u}{t} \odif{x} & = \left[ -\kappa \pdv{u}{x} \right]_{x_1} - \left[ -\kappa \pdv{u}{x} \right]_{x_2} \\
    \int_{x_1}^{x_2} \rho c \pdv{u}{t} \odif{x} & = \left[ \kappa \pdv{u}{x} \right]_{x_2} - \left[ \kappa \pdv{u}{x} \right]_{x_1}   \\
    \int_{x_1}^{x_2} \rho c \pdv{u}{t} \odif{x} & = \int_{x_1}^{x_2} \pdv*{\kappa \pdv{u}{x}}{x} \odif{x}                             \\
    \rho c \pdv{u}{t}                           & = \pdv*{\kappa \pdv{u}{x}}{x}                                                       \\
    \pdv{u}{t}                                  & = \frac{\kappa}{\rho c} \pdv[order=2]{u}{x}
\end{align*}
where \(k\) is the thermal diffusivity of the rod:
\begin{equation*}
    \pdv{u}{t} = k\pdv[order=2]{u}{x}.
\end{equation*}
More generally, we can write the PDE as
\begin{equation*}
    \pdv{\symbfit{u}}{t} = k \Delta{\symbfit{u}}
\end{equation*}
for multiple spatial dimensions. This PDE is called the heat equation.
The heat equation is first order w.r.t.\ time and second order w.r.t.\ space.
\subsection{Wave Equation}
Consider an elastic string that is stretched tightly with its two ends
fixed at \(x = 0\) and \(x = L\) where the vertical displacement of the
string is given by \(u\left( x,\: t \right)\), and the initial
displacement is arbitrary: \(u\left( x,\: 0 \right) = f\left( x
\right)\). Let \(\theta\left( x,\: t \right)\) be the angle of the
string from the horizontal with tension \(T\left( x,\: t \right)\)
(magnitude). We can then apply the law of conservation. In the
horizontal direction, assume equilibrium:
\begin{equation*}
    T\left( x_1,\: t \right) \cos{\left( \theta\left( x_1,\: t \right) \right)} = T\left( x_2,\: t \right) \cos{\left( \theta\left( x_2,\: t \right) \right)}
\end{equation*}
In the vertical direction, assume no external forces:
\begin{align*}
    m a                                                & = \sum F                                                                                                                                                     \\
    \int_{x_1}^{x_2} \rho \pdv[order=2]{u}{t} \odif{S} & = -T\left( x_1,\: t \right) \sin{\left( \theta\left( x_1,\: t \right) \right)} + T\left( x_2,\: t \right) \sin{\left( \theta\left( x_2,\: t \right) \right)}
\end{align*}
where \(\rho\) is the linear density of the string, and the integral is defined along the arc \(\odif{S}\).
If we assume that the magnitude of the rate of displacement is small, then
\begin{gather*}
    \theta \approx \sin{\left( \theta \right)} \approx \tan{\left( \theta \right)} = \pdv{u}{x} \\
    \cos{\left( \theta \right)} \approx 1
\end{gather*}
therefore in the horizontal direction,
\begin{equation*}
    T\left( x_1,\: t \right) = T\left( x_2,\: t \right)
\end{equation*}
the tension is independent of \(x\). In the vertical direction,
\begin{align*}
    \int_{x_1}^{x_2} \rho \pdv[order=2]{u}{t} \sqrt{1 + \left( \pdv{u}{x} \right)^2} \odif{x} & = -T\left( x_1,\: t \right) \left[ \pdv{u}{x} \right]_{x_1} + T\left( x_2,\: t \right) \left[ \pdv{u}{x} \right]_{x_2} \\
    \int_{x_1}^{x_2} \rho \pdv[order=2]{u}{t} \odif{x}                                        & = T \int_{x_1}^{x_2} \pdv[order=2]{u}{x} \odif{x}                                                                      \\
    \pdv[order=2]{u}{t}                                                                       & = \frac{T}{\rho} \pdv[order=2]{u}{x}                                                                                   \\
    \pdv[order=2]{u}{t}                                                                       & = c^2 \pdv[order=2]{u}{x}
\end{align*}
where \(c = \sqrt{\frac{T}{\rho}}\) is known as the \textit{wave speed}. This PDE is known as the wave equation.
As this PDE is second order w.r.t.\ time, the second initial condition is
\begin{equation*}
    \pdv{u}{t}\left( x,\: 0 \right) = g\left( x \right)
\end{equation*}
where \(g\left( x \right)\) is an initial velocity applied to the string.
\subsection{Laplace's Equation}
By considering higher spatial dimensions, we can model the temperature
of a plate \(u\left( x,\: y,\: t \right)\) with:
\begin{equation*}
    \pdv{u}{t} = k \left( \pdv[order=2]{u}{x} + \pdv[order=2]{u}{y} \right)
\end{equation*}
and similarly the displacement of an elastic membrane \(u\left( x,\: y,\: t \right)\):
\begin{equation*}
    \pdv[order=2]{u}{t} = c^2 \left( \pdv[order=2]{u}{x} + \pdv[order=2]{u}{y} \right).
\end{equation*}
The time-independent or \textbf{steady-state} case of these equations yields
\begin{equation*}
    \pdv[order=2]{u}{x} + \pdv[order=2]{u}{y} = 0
\end{equation*}
which is known as Laplace's equation. Commonly, this equation is written using the Laplacian operator,
\begin{equation*}
    \Delta{u} = \nabla^2 u = \nabla \cdot \nabla u = 0.
\end{equation*}
When this equation is nonhomogeneous, the PDE is known as Poisson's equation.
\subsection{Classification of Linear Second Order PDEs}
All second order, linear partial differential equations in two
dimensions (either space and time or space and space) may be written in
the following way:
\begin{equation*}
    A\left( x,\: y \right) \pdv[order=2]{u}{x} + B\left( x,\: y \right) \pdv{u}{x,y} + C\left( x,\: y \right) \pdv[order=2]{u}{y} + D\left( x,\: y \right) \pdv{u}{x} + E\left( x,\: y \right) \pdv{u}{y} + F\left( x,\: y \right) u = G\left( x,\: y \right).
\end{equation*}
We classify the equation as follows:
\begin{itemize}
    \item Hyperbolic: \(B^2 - 4A C > 0\),
    \item Parabolic: \(B^2 - 4A C = 0\),
    \item Elliptical: \(B^2 - 4A C < 0\).
\end{itemize}
It follows that the heat equation is parabolic, the wave equation is hyperbolic and the Laplace equation is elliptical.
\section{Separation of Variables}
To solve an IBVP consider the following:
\begin{enumerate}
    \item Assume a set of solutions of the form
          \begin{equation*}
              u_n\left( x,\: t \right) = X_n\left( x \right) T_n\left( t \right).
          \end{equation*}
    \item Substitute \(u_n\) into the homogeneous PDE and separate
          \begin{equation*}
              f_1\left( x,\: X,\: X',\: \dots \right) = f_2\left( t,\: T,\: T',\: \dots \right).
          \end{equation*}
    \item As each term depends on a different variable, each \(f_i\)
          must be a scalar \(\alpha_n\).
          \begin{align*}
              f_1\left( x,\: X,\: X',\: \dots \right) & = \alpha_n  \\
              f_2\left( t,\: T,\: T',\: \dots \right) & = \alpha_n.
          \end{align*}
    \item Solve the ODEs with boundary conditions while selecting
          appropriate values of \(\alpha_n\) (i.e., negative, zero,
          positive) that produce non-trivial solutions.
    \item Solve the remaining ODEs using \(\alpha_n\) from the previous
          step.
    \item Use the principle of superposition to construct a general
          solution:
          \begin{equation*}
              u\left( x,\: t \right) = \sum_{n = 1}^\infty u_n\left( x,\: t \right).
          \end{equation*}
    \item Calculate any remaining constants using initial conditions.
\end{enumerate}
\subsection{Separation of Variables: Heat Equation}
Assuming the following conditions:
\begin{equation*}
    u\left( x,\: 0 \right) = f\left( x \right), \quad\quad u\left( 0,\: t \right) = u\left( L,\: t \right) = 0.
\end{equation*}
\begin{align*}
    \pdv{u}{t}               & = k\pdv[order=2]{u}{x}     \\
    \pdv{XT}{t}              & = k\pdv[order=2]{XT}{x}    \\
    XT'                      & = kX''T                    \\
    \frac{1}{k} \frac{T'}{T} & = \frac{X''}{X} = \alpha_n
\end{align*}
This results in the following two ODEs
\begin{align*}
    T' - \alpha_n k T & = 0 \\
    X'' - \alpha_n X  & = 0
\end{align*}
\subsubsection{Spatial Dimension}
\begin{proofcase}{1. \(\alpha_n > 0\)}\let\qed\relax
    \begin{align*}
        m^2 - \alpha_n & = 0                     \\
        m              & = \pm \sqrt{ \alpha_n }
    \end{align*}
    Therefore
    \begin{equation*}
        X_n\left( x \right) = c_1 e^{\sqrt{ \alpha_n } x} + c_2 e^{-\sqrt{ \alpha_n } x}.
    \end{equation*}
    Applying the BCs gives
    \begin{align*}
        X_n\left( 0 \right) & = c_1 + c_2 = 0                                              \\
        X_n\left( L \right) & = c_1 e^{\sqrt{\alpha_n} L} + c_2 e^{-\sqrt{\alpha_n} L} = 0
    \end{align*}
    so that
    \begin{equation*}
        \begin{bmatrix*}
            1 & 1 \\
            e^{\sqrt{\alpha_n} L} & e^{-\sqrt{\alpha_n} L}
        \end{bmatrix*}
        \begin{bmatrix*}
            c_1 \\
            c_2
        \end{bmatrix*}
        =
        \begin{bmatrix*}
            0 \\
            0
        \end{bmatrix*}
    \end{equation*}
    This homogeneous equation has non-trivial solutions iff the determinant is zero.
    \begin{align*}
        \begin{vmatrix*}
            1 & 1 \\
            e^{\sqrt{\alpha_n} L} & e^{-\sqrt{\alpha_n} L}
        \end{vmatrix*}
                                                       & = 0 \\
        e^{-\sqrt{\alpha_n} L} - e^{\sqrt{\alpha_n} L} & = 0 \\
        -2\sinh{\left( \sqrt{\alpha_n} L \right)}      & = 0 \\
        \alpha_n                                       & = 0
    \end{align*}
    but as \(\alpha_n > 0\), no solutions exist.
\end{proofcase}
\begin{proofcase}{2. \(\alpha_n = 0\)}\let\qed\relax
    \begin{equation*}
        X_n\left( x \right) = c_1 x + c_2.
    \end{equation*}
    Applying the BCs gives
    \begin{align*}
        X_n\left( 0 \right) & = c_2 = 0                                   \\
        X_n\left( L \right) & = c_1 \left( L \right) = 0 \implies c_1 = 0
    \end{align*}
    hence there are no non-trivial solutions as \(X_n \equiv 0\).
\end{proofcase}
\begin{proofcase}{3. \(\alpha_n < 0\)}\let\qed\relax
    \begin{align*}
        m^2 + \alpha_n & = 0                      \\
        m              & = \pm \sqrt{-\alpha_n} i
    \end{align*}
    therefore
    \begin{equation*}
        X_n\left( x \right) = c_1 \cos{\left( \sqrt{-\alpha_n} x \right)} + c_2 \sin{\left( \sqrt{-\alpha_n} x \right)}.
    \end{equation*}
    Applying the BCs gives
    \begin{align*}
        X_n\left( 0 \right) & = c_1 = 0                                         \\
        X_n\left( L \right) & = c_2 \sin{\left( \sqrt{-\alpha_n} L \right)} = 0
    \end{align*}
    therefore
    \begin{align*}
        \sqrt{-\alpha_n} L & = n \pi                  \\
        \alpha_n           & = -\frac{n^2 \pi^2}{L^2}
    \end{align*}
    which gives the following family of solutions:
    \begin{equation*}
        X_n\left( x \right) = c_2 \sin{\left( \frac{n \pi}{L} x \right)}
    \end{equation*}
\end{proofcase}
\subsubsection{Time Dimension}
\begin{align*}
    m - \alpha_n k & = 0          \\
    m              & = \alpha_n k
\end{align*}
which gives
\begin{equation*}
    T_n\left( t \right) = c_3 e^{\alpha_n k t} = c_3 e^{-\frac{n^2 \pi^2}{L^2} k t}.
\end{equation*}
\subsubsection{General Solution}
Given these two functions, we can solve for \(u_n\) as
\begin{align*}
    u_n\left( x,\: t \right) = B_n \sin{\left( \frac{n \pi}{L} x \right)} e^{-\frac{n^2 \pi^2}{L^2} k t}
\end{align*}
then by applying superposition, we find the general solution to the PDE\@:
\begin{equation*}
    u\left( x,\: t \right) = \sum_{n = 1}^\infty u_n\left( x,\: t \right) = \sum_{n = 1}^\infty B_n \sin{\left( \frac{n \pi}{L} x \right)} e^{-\frac{n^2 \pi^2}{L^2} k t}.
\end{equation*}
Applying the initial conditions gives
\begin{align*}
    u\left( x,\: 0 \right) = \sum_{n = 1}^\infty B_n \sin{\left( \frac{n \pi}{L} x \right)} = f\left( x \right)
\end{align*}
so that the coefficients \(B_n\) are given by the Fourier sine coefficients of the initial condition \(f\left( x \right)\).
Therefore, the general solution to the PDE is given by
\begin{equation*}
    u\left( x,\: t \right) = \sum_{n = 1}^\infty B_n \sin{\left( \frac{n \pi}{L} x \right)} e^{-\frac{n^2 \pi^2}{L^2} k t}, \quad \quad B_n = \frac{2}{L} \int_0^L f\left( x \right) \sin{\left( \frac{n \pi}{L} x \right)} \odif{x}.
\end{equation*}
In this solution, as time tends to infinity, the exponential forces the solution to tend toward \(0\).
We also observe that for large \(n\), the sum produces very small values, and hence we can say
\begin{equation*}
    u\left( x,\: t \right) \approx B_1 \sin{\left( \frac{\pi}{L} x \right)} e^{-\frac{\pi^2}{L^2} k t} .
\end{equation*}
For large \(t\)
\begin{equation*}
    u\left( x,\: t \right) \approx B_1 \sin{\left( \frac{\pi}{L} x \right)}.
\end{equation*}
\subsection{Separation of Variables: Wave Equation}
Assume that the initial velocity is 0 and that the ends of the string
can move freely in the direction of the string, so that the conditions
are given by
\begin{equation*}
    u\left( x,\: 0 \right) = f\left( x \right), \quad \quad \pdv{u}{t} \left( x,\: 0 \right) = 0, \quad \quad \pdv{u}{x} \left( 0,\: t \right) = \pdv{u}{x} \left( L,\: t \right) = 0.
\end{equation*}
Then by using the ansatz
\begin{align*}
    \pdv[order=2]{u}{t}         & = c^2\pdv[order=2]{u}{x}   \\
    \pdv[order=2]{XT}{t}        & = c^2\pdv[order=2]{XT}{x}  \\
    XT''                        & = c^2X''T                  \\
    \frac{1}{c^2} \frac{T''}{T} & = \frac{X''}{X} = \alpha_n
\end{align*}
This results in the following two ODEs
\begin{align*}
    T'' - \alpha_n c^2 T & = 0 \\
    X'' - \alpha_n X     & = 0
\end{align*}
\subsubsection{Spatial Dimension}
\begin{proofcase}{1. \(\alpha_n > 0\)}\let\qed\relax
    \begin{align*}
        m^2 - \alpha_n & = 0                     \\
        m              & = \pm \sqrt{ \alpha_n }
    \end{align*}
    Therefore
    \begin{equation*}
        X_n\left( x \right) = c_1 e^{\sqrt{ \alpha_n } x} + c_2 e^{-\sqrt{ \alpha_n } x}
    \end{equation*}
    with
    \begin{equation*}
        X_n'\left( x \right) = c_1 \sqrt{ \alpha_n } e^{\sqrt{ \alpha_n } x} - c_2 \sqrt{ \alpha_n } e^{-\sqrt{ \alpha_n } x}.
    \end{equation*}
    Applying the BCs gives
    \begin{align*}
        X_n'\left( 0 \right) & = c_1 \sqrt{\alpha_n} - c_2 \sqrt{\alpha_n} = 0                                              \\
        X_n'\left( L \right) & = c_1 \sqrt{\alpha_n} e^{\sqrt{\alpha_n} L} - c_2 \sqrt{\alpha_n} e^{-\sqrt{\alpha_n} L} = 0
    \end{align*}
    so that
    \begin{equation*}
        \begin{bmatrix*}
            \sqrt{\alpha_n} & -\sqrt{\alpha_n} \\
            \sqrt{\alpha_n} e^{\sqrt{\alpha_n} L} & -\sqrt{\alpha_n} e^{-\sqrt{\alpha_n} L}
        \end{bmatrix*}
        \begin{bmatrix*}
            c_1 \\
            c_2
        \end{bmatrix*}
        =
        \begin{bmatrix*}
            1 & -1 \\
            e^{\sqrt{\alpha_n} L} & -e^{-\sqrt{\alpha_n} L}
        \end{bmatrix*}
        \begin{bmatrix*}
            c_1 \\
            c_2
        \end{bmatrix*}
        =
        \begin{bmatrix*}
            0 \\
            0
        \end{bmatrix*}
    \end{equation*}
    This homogeneous equation has non-trivial solutions iff the determinant is zero.
    \begin{align*}
        \begin{vmatrix*}
            1 & -1 \\
            e^{\sqrt{\alpha_n} L} & -e^{-\sqrt{\alpha_n} L}
        \end{vmatrix*}
                                                        & = 0 \\
        -e^{-\sqrt{\alpha_n} L} + e^{\sqrt{\alpha_n} L} & = 0 \\
        2\sinh{\left( \sqrt{\alpha_n} L \right)}        & = 0 \\
        \alpha_n                                        & = 0
    \end{align*}
    but as \(\alpha_n > 0\), no solutions exist.
\end{proofcase}
\begin{proofcase}{2. \(\alpha_n = 0\)}\let\qed\relax
    \begin{equation*}
        X_n\left( x \right) = c_1 x + c_2
    \end{equation*}
    with
    \begin{equation*}
        X_n'\left( x \right) = c_1.
    \end{equation*}
    Applying the BCs gives
    \begin{align*}
        X_n'\left( 0 \right) & = 0 = 0   \\
        X_n'\left( L \right) & = c_1 = 0
    \end{align*}
    therefore
    \begin{equation*}
        X_n\left( x \right) = c_2
    \end{equation*}
    is a solution.
\end{proofcase}
\begin{proofcase}{3. \(\alpha_n < 0\)}\let\qed\relax
    \begin{align*}
        m^2 + \alpha_n & = 0                      \\
        m              & = \pm \sqrt{-\alpha_n} i
    \end{align*}
    therefore
    \begin{equation*}
        X_n\left( x \right) = c_1 \cos{\left( \sqrt{-\alpha_n} x \right)} + c_2 \sin{\left( \sqrt{-\alpha_n} x \right)}
    \end{equation*}
    with
    \begin{equation*}
        X_n'\left( x \right) = -c_1 \sqrt{-\alpha_n} \sin{\left( \sqrt{-\alpha_n} x \right)} + c_2 \sqrt{-\alpha_n} \cos{\left( \sqrt{-\alpha_n} x \right)}.
    \end{equation*}
    Applying the BCs gives
    \begin{align*}
        X_n'\left( 0 \right) & = c_2 \sqrt{-\alpha_n} = 0 \implies c_2 = 0                         \\
        X_n'\left( L \right) & = -c_1 \sqrt{-\alpha_n} \sin{\left( \sqrt{-\alpha_n} L \right)} = 0
    \end{align*}
    therefore
    \begin{align*}
        \sqrt{-\alpha_n} L & = n \pi                  \\
        \alpha_n           & = -\frac{n^2 \pi^2}{L^2}
    \end{align*}
    which gives the following family of solutions:
    \begin{equation*}
        X_n\left( x \right) = c_1 \cos{\left( \frac{n \pi}{L} x \right)}
    \end{equation*}
\end{proofcase}
\subsubsection{Time Dimension}
As we found two cases for \(\alpha_n\), we must do the same for
\(T_n\).
\begin{proofcase}{1. \(\alpha_n < 0\)}\let\qed\relax
    \begin{align*}
        m^2 - \alpha_n c^2 & = 0                        \\
        m^2                & = \alpha_n c^2             \\
        m                  & = \pm \sqrt{\alpha_n} c    \\
        m                  & = \pm \sqrt{-\alpha_n} c i
    \end{align*}
    which gives
    \begin{equation*}
        T_n\left( t \right) = c_3 \cos{\left( \sqrt{-\alpha_n} c t \right)} + c_4 \sin{\left( \sqrt{-\alpha_n} c t \right)} = c_3 \cos{\left( \frac{n \pi}{L} c t \right)} + c_4 \sin{\left( \frac{n \pi}{L} c t \right)}.
    \end{equation*}
\end{proofcase}
\begin{proofcase}{2. \(\alpha_n = 0\)}\let\qed\relax
    \begin{align*}
        m^2 & = 0 \\
        m   & = 0
    \end{align*}
    which gives
    \begin{equation*}
        T_n\left( t \right) = c_3 t + c_4.
    \end{equation*}
\end{proofcase}
\subsubsection{General Solution}
Given these two functions, we find two solutions for \(u_n\)
\begin{align*}
    u_n\left( x,\: t \right) = \cos{\left( \frac{n \pi}{L} x \right)} \left[ A_n \cos{\left( \frac{n \pi}{L} c t \right)} + B_n \sin{\left( \frac{n \pi}{L} c t \right)} \right].
\end{align*}
for \(\alpha_n < 0\), and also
\begin{align*}
    u_0\left( x,\: t \right) = A_0 + B_0 t
\end{align*}
for \(\alpha_n = 0\), where \(u_0\) does not depend on \(n\).
By applying superposition, we find the general solution to the PDE\@:
\begin{equation*}
    u\left( x,\: t \right) = u_0\left( x,\: t \right) + \sum_{n = 1}^\infty u_n\left( x,\: t \right) = A_0 + B_0 t + \sum_{n = 1}^\infty \cos{\left( \frac{n \pi}{L} x \right)} \left[ A_n \cos{\left( \frac{n \pi}{L} c t \right)} + B_n \sin{\left( \frac{n \pi}{L} c t \right)} \right].
\end{equation*}
Applying the initial conditions gives
\begin{align*}
    u_n\left( x,\: 0 \right) = A_0 + \sum_{n = 1}^\infty A_n \cos{\left( \frac{n \pi}{L} x \right)} = f\left( x \right)
\end{align*}
so that the coefficients \(A_n\) are given by the Fourier cosine coefficients of the initial condition \(f\left( x \right)\).
Applying the second initial condition requires the first derivative
w.r.t.\ \(x\):
\begin{equation*}
    \pdv{u\left( x,\: t \right)}{x} = B_0 + \sum_{n = 1}^\infty \frac{n \pi}{L} c \cos{\left( \frac{n \pi}{L} x \right)} \left[ B_n \cos{\left( \frac{n \pi}{L} c t \right)} - A_n \sin{\left( \frac{n \pi}{L} c t \right)} \right]
\end{equation*}
so that
\begin{equation*}
    \pdv{u}{x}\left( x,\: 0 \right) = B_0 + \sum_{n = 1}^\infty \frac{n \pi}{L} c B_n \cos{\left( \frac{n \pi}{L} x \right)} = 0.
\end{equation*}
In this case, a zero initial velocity requires \(B_0 = B_n = 0\).
Therefore, the solution to the IBVP is given by
\begin{equation*}
    u\left( x,\: t \right) = A_0 + \sum_{n = 1}^\infty A_n \cos{\left( \frac{n \pi}{L} x \right)} \cos{\left( \frac{n \pi}{L} c t \right)},
\end{equation*}
where
\begin{equation*}
    A_0 = \frac{1}{2L} \int_{-L}^L f\left( x \right) \odif{x}, \quad\quad A_n = \frac{2}{L} \int_{-L}^{L} f\left( x \right) \cos{\left( \frac{n \pi}{L} x \right)} \odif{x}.
\end{equation*}
\subsection{Separation of Variables: Laplace's Equation}
Let us assume the following boundary conditions for Laplace's equation:
\begin{equation*}
    u\left( x,\: 0 \right) = 0, \quad \quad u\left( x,\: 1 \right) = x^2, \quad \quad u\left( 0,\: y \right) = u\left( 1,\: y \right) = 0
\end{equation*}
so that our region of interest is given by the unit square.
Then by using the ansatz
\begin{align*}
    \pdv[order=2]{u}{x} + \pdv[order=2]{u}{y}   & = 0                          \\
    \pdv[order=2]{XY}{x} + \pdv[order=2]{XY}{y} & = 0                          \\
    X''Y + XY''                                 & = 0                          \\
    \frac{X''}{X}                               & = - \frac{Y''}{Y} = \alpha_n
\end{align*}
This results in the following two ODEs
\begin{align*}
    X'' - \alpha_n X & = 0 \\
    Y'' + \alpha_n Y & = 0
\end{align*}
\subsubsection{Problem for \texorpdfstring{\(X\)}{X}}
Let us first consider the problem for \(X\) as a boundary condition for
\(Y\) is nonhomogeneous.
From the heat equation, we know that the only nontrivial solutions to
this ODE occur when
\begin{equation*}
    \alpha_n = - n^2 \pi^2 < 0, \quad \quad X\left( x \right) = X_n\left( x \right) = c \sin{\left( n \pi x \right)}, \quad \quad n = 1,\: 2,\: 3,\: \dots
\end{equation*}
for constant \(c\).
\subsubsection{Problem for \texorpdfstring{\(Y\)}{Y}}
The problem for \(Y\) yields the following solution:
\begin{equation*}
    Y\left( y \right) = Y_n\left( y \right) = A_n \cosh{\left( n \pi y \right)} + B_n \sinh{\left( n \pi y \right)}.
\end{equation*}
\subsubsection{General Solution}
Given these two functions, we can use superposition to find
\begin{align*}
    u\left( x,\: y \right) = \sum_{n = 1}^\infty X_n\left( x \right) Y_n\left( y \right) = \sum_{n = 1}^\infty \left[ A_n \cosh{\left( n \pi y \right)} + B_n \sinh{\left( n \pi y \right)} \right] \sin{\left( n \pi x \right)}.
\end{align*}
We can now apply the boundary conditions in \(y\). At \(y = 0\)
\begin{align*}
    u\left( x,\: 0 \right) = \sum_{n = 1}^\infty A_n \sin{\left( n \pi x \right)} = 0 \implies A_n = 0
\end{align*}
At \(y = 1\):
\begin{align*}
    u\left( x,\: 1 \right) = \sum_{n = 1}^\infty B_n \sinh{\left( n \pi \right)} \sin{\left( n \pi x \right)} = x^2
\end{align*}
Here we can use the sine series expansion of \(x^2\) where the coefficient is now multiplied by \(\sinh{\left( n \pi \right)}\).
Therefore,
\begin{equation*}
    u\left( x,\: y \right) = \sum_{n = 1}^\infty B_n \sinh{\left( n \pi y \right)} \sin{\left( n \pi x \right)}
\end{equation*}
with
\begin{equation*}
    B_n \sinh{\left( n \pi \right)} = 2 \int_0^1 x^2 \sin{\left( n \pi x \right)} \odif{x}.
\end{equation*}
\section{Sturm-Liouville Theory}
Sturm-Liouville theory is used to solve real second-order linear ODEs
of the form:
\begin{equation*}
    \odv{}{x} \left[ p\left( x \right) \odv{y}{x} \right] + q\left( x \right) y + \lambda w\left( x \right) y = 0,
\end{equation*}
with the boundary conditions
\begin{align*}
    -l_1 y'\left( a \right) + h_1 y\left( a \right) & = 0 \\
    l_2 y'\left( b \right) + h_2 y\left( b \right)  & = 0
\end{align*}
where both boundary conditions must be non-trivial (\(l\) or \(h\) is non-zero). A Sturm-Liouville problem is
\textbf{regular} when \(p\left( x \right), w\left( x \right) > 0\), and \(p\left( x \right), p'\left( x \right), q\left( x \right), w\left( x \right)\) are continuous
over the interval \(\interval{a}{b}\).
A second-order ODE of the form
\begin{equation*}
    a_2\left( x \right) y''\left( x \right) + a_1\left( x \right) y'\left( x \right) + a_0\left( x \right) y\left( x \right) = \lambda y\left( x \right)
\end{equation*}
can be converted into SL form by multiplying the ODE by the integrating factor
\begin{equation*}
    \mu = \frac{1}{a_2} \exp{\left( \int \frac{a_1}{a_2} \odif{x} \right)}.
\end{equation*}
\subsection{Weighted Inner Product}
The function \(w\left( x \right) > 0\) is known as the \textbf{weight
function} with which we can define the inner product:
\begin{equation*}
    \abracket*{f,\: g}_{w} = \int_a^b f\left( x \right) g\left( x \right) w\left( x \right) \odif{x}.
\end{equation*}
\subsection{Eigenvalue Problem}
By defining the mapping:
\begin{equation*}
    u \mapsto - \frac{1}{w\left( x \right)} \left( \odv{}{x} \left[ p\left( x \right) \odv{u}{x} \right] + q\left( x \right) u \right)
\end{equation*}
with the linear operator \(L\), we can consider the associated eigenvalue problem of
the Sturm-Liouville system:
\begin{equation*}
    L u = \lambda u.
\end{equation*}
\subsection{Self-Adjointness}
Here we recognise that \(L\) is a \textbf{self-adjoint} operator, such
that:
\begin{equation*}
    \abracket*{Lf,\: g}_{w} = \abracket*{f,\: Lg}_{w}.
\end{equation*}
\subsection{Orthogonality}
It then follows that all solutions to this ODE produce an infinite
number of real \textbf{eigenvalues} \(\lambda_i\), where \(\lambda_n
\to \infty\) as \(n \to \infty\), where the corresponding
\textbf{eigenfunctions} \(u_i\) of \(L\) are \textbf{orthogonal} with
respect to the weighted inner product. Taking the weighted inner
product between normalised eigenfunctions shows that
\begin{equation*}
    \abracket*{y_n,\: y_m} = \int_a^b y_n\left( x \right) y_m\left( x \right) w\left( x \right) \odif{x} = \delta_{mn}
\end{equation*}
where \(\delta_{mn}\) is the Kronecker delta.
\subsection{Sign of Eigenvalues}
A \textbf{proper} Sturm-Liouville system is a system in which \(q\left(
x \right) \leqslant 0\) on \(\interval{a}{b}\), with \(l_1h_1 \geqslant
0\) and \(l_2h_2 \geqslant 0\). All eigenvalues of a proper
Sturm-Liouville system are non-negative.
\subsection{Singular and Periodic Sturm-Liouville Systems}
\begin{itemize}
    \item When \(p\left( a \right) = 0\), and the BC at \(x = a\) is
          replaced by the condition that \(y\) remain bounded; the
          system is \textbf{singular}\footnote{The same applies with
          the boundary condition at \(x = b\)}.
    \item If instead of the BCs we have:
          \begin{equation*}
              p\left( a \right) = p\left( b \right) \quad \text{and} \quad p'\left( a \right) = p'\left( b \right)
          \end{equation*}
          then we have a \textbf{periodic} system, where \(y\) must also be periodic.
\end{itemize}
\subsection{Eigenfunction Expansions}
If we treat the set of eigenfunctions of a Sturm-Liouville system as a
\textbf{basis}, we can write a given function \(f\) as a linear
combination of eigenfunctions. Given an orthogonal basis \(\left\{
y_n\left( x \right) : n \in \Z^{+} \right\}\), the eigenfunction
expansion of \(f\) is given by
\begin{equation*}
    f_E\left( x \right) = \sum_{n = 1}^\infty c_n y_n\left( x \right)
\end{equation*}
with
\begin{equation*}
    c_n = \frac{\abracket*{f_E,\: y_n}_w}{\abracket*{y_n,\: y_n}_w} = \frac{\abracket*{f_E,\: y_n}_w}{\norm*{y_n}^2}.
\end{equation*}
where the usual definition of the norm applies:
\begin{equation*}
    \norm*{y_n} = \sqrt{\abracket*{y_n,\: y_n}}.
\end{equation*}
To prove this, consider the inner product of the function \(f\) with
a particular eigenfunction \(y_m\):
\begin{align*}
    f\left( x \right)      & = \sum_{n = 1}^\infty c_n y_n\left( x \right)              \\
    \abracket*{f,\: y_m}_w & = \sum_{n = 1}^\infty c_n \abracket*{y_n,\: y_m}_w         \\
    \abracket*{f,\: y_m}_w & = c_m \abracket*{y_m,\: y_m}_w                             \\
    c_m                    & = \frac{\abracket*{f,\: y_m}_w}{\abracket*{y_m,\: y_m}_w}.
\end{align*}
This result generalises the Fourier series expansion introduced in Section 1, as our basis
is no longer restricted to trigonometric functions.
\subsubsection{Convergence}
As with Fourier series, \(f_E\) does not necessarily converge to \(f\).
For instance, if \(f\) is piecewise smooth,
\begin{equation*}
    f_E\left( x \right) = \lim_{\epsilon \to 0^+} \frac{f\left( x + \epsilon \right) + f\left( x - \epsilon \right)}{2}
\end{equation*}
for \(a < x < b\).
\section{Polar Coordinates}
When considering problems posed on circular regions, we can apply
coordinate transformation to solve problems using polar coordinates.
Recall that we can write \(x\) and \(y\) in terms of \(r\) and
\(\theta\):
\begin{equation*}
    x = r \cos{\left( \theta \right)}, \quad y = r \sin{\left( \theta \right)} \quad \iff \quad r = \sqrt{x^2 + y^2}, \quad \theta = \arctan{\left( \frac{y}{x} \right)}.
\end{equation*}
This allows us to express partial derivatives of \(u\left( x, \: y \right)\) in terms of \(r\) and \(\theta\) using the multivariable chain rule:
\begin{align*}
    \pdv{u}{x} & = \pdv{u}{r} \pdv{r}{x} + \pdv{u}{\theta} \pdv{\theta}{x}                                        \\
               & = \frac{2x}{\sqrt{x^2 + y^2}} \pdv{u}{r} + \frac{-y/x^2}{1+\left( y/x \right)^2} \pdv{u}{\theta} \\
               & = \frac{2x}{r} \pdv{u}{r} + \frac{-y}{r^2} \pdv{u}{\theta}
\end{align*}
and
\begin{align*}
    \pdv{u}{y} & = \pdv{u}{r} \pdv{r}{y} + \pdv{u}{\theta} \pdv{\theta}{y}                                     \\
               & = \frac{2y}{\sqrt{x^2 + y^2}} \pdv{u}{r} + \frac{1/x}{1+\left( y/x \right)^2} \pdv{u}{\theta} \\
               & = \frac{2y}{r} \pdv{u}{r} + \frac{x}{r^2} \pdv{u}{\theta}.
\end{align*}
\subsection{The Laplacian in Polar Coordinates}
Recall Laplace's equation in Cartesian coordinates
\begin{equation*}
    \nabla^2 u = \pdv[order=2]{u}{x} + \pdv[order=2]{u}{y}.
\end{equation*}
By using the multivariable chain rule, we can express the Laplacian in terms of \(r\) and \(\theta\):
\begin{equation*}
    \nabla^2 u = \pdv[order=2]{u}{r} + \frac{1}{r} \pdv{u}{r} + \frac{1}{r^2} \pdv[order=2]{u}{\theta}.
\end{equation*}
\subsection{Laplace's Equation in Polar Coordinates}
Consider the following example of Laplace's equation on a disk of
radius \(a\):
\begin{equation*}
    \pdv[order=2]{u}{r} + \frac{1}{r} \pdv{u}{r} + \frac{1}{r^2} \pdv[order=2]{u}{\theta} = 0, \quad\quad 0 \leqslant r < a, -\pi < \theta \leqslant \pi,
\end{equation*}
for \(u = u\left( r,\: \theta \right)\) with the following boundary condition on \(r = a\):
\begin{equation*}
    u\left( a,\: \theta \right) = f\left( \theta \right).
\end{equation*}
Here consider a solution of the form
\begin{equation*}
    u\left( r,\: \theta \right) = R\left( r \right) \Theta\left( \theta \right)
\end{equation*}
so that
\begin{align*}
    u_{r r} + \frac{1}{r} u_r + \frac{1}{r^2} u_{\theta\theta}                       & = 0                                      \\
    R''\Theta + \frac{1}{r} R'\Theta + \frac{1}{r^2} R\Theta''                       & = 0                                      \\
    \frac{R''}{R} + \frac{1}{r} \frac{R'}{R} + \frac{1}{r^2} \frac{\Theta''}{\Theta} & = 0                                      \\
    \frac{R''}{R} + \frac{1}{r} \frac{R'}{R}                                         & = -\frac{1}{r^2} \frac{\Theta''}{\Theta} \\
    r^2 \frac{R''}{R} + r \frac{R'}{R}                                               & = -\frac{\Theta''}{\Theta} = - \alpha
\end{align*}
therefore
\begin{align*}
    r^2 R'' + r R' + \alpha R & = 0  \\
    \Theta'' - \alpha \Theta  & = 0.
\end{align*}
\subsubsection{Periodicity}
To identify the Sturm-Liouville problem, we require a homogeneous
equation with homogeneous boundary conditions. As \(u\) is defined on a
circular disk, it must satisfy the following condition of periodicity:
\begin{equation*}
    u\left( r,\: \theta \right) = u\left( r,\: \theta + 2\pi \right).
\end{equation*}
Additionally, as \(u\) is defined in terms of two separable functions,
\(u = R \Theta\), this periodicity must also hold in \(\Theta\):
\begin{equation*}
    \Theta\left( \theta \right) = \Theta\left( \theta + 2\pi \right).
\end{equation*}
Solving for \(\Theta\) reveals that only \(\alpha_n \leqslant 0\) has
periodic solutions, therefore for \(n = 1, 2, \ldots\) we have
\begin{equation*}
    \Theta_0 = 1, \quad\quad \Theta_{n1} = \cos{\left( n\theta \right)}, \quad\quad \Theta_{n2} = \sin{\left( n\theta \right)}.
\end{equation*}
where \(\alpha_n = -n^2\). Therefore, unlike the previous examples, each eigenvalue
has two linearly independent eigenfunctions. Solving the problem in \(R\) yields
the Cauchy-Euler equation:
\begin{equation*}
    r^2 R'' + r R' - n^2 R = 0.
\end{equation*}
By assuming \(R = r^m\), we find \(m = \pm n\). If we consider positive values of \(n\):
\begin{equation*}
    R_n = c_1 r^n + c_2 r^{-n}.
\end{equation*}
However, when \(n = 0\),
\begin{equation*}
    R_0 = c_1 \ln{\left( r \right)} + c_2.
\end{equation*}
\subsubsection{Boundedness}
As \(u\) is defined on a circular disk, it must be bounded at \(r =
0\). Therefore, \(R_n\) must be bounded at \(r = 0\), which implies
that in the first solution, \(c_2 = 0\), and in the second solution,
\(c_1 = 0\). This results in
\begin{equation*}
    R_0 = c_2, \quad\quad R_n = c_1 r^n.
\end{equation*}
By applying superposition,
\begin{equation*}
    u\left( r,\: \theta \right) = A_0 + \sum_{n = 1}^\infty \left[ A_n r_n \cos{\left( n \theta \right)} + B_n r_n \sin{\left( n \theta \right)} \right].
\end{equation*}
\subsubsection{Boundary Conditions}
Assuming \(u\left( a,\: \theta \right) = f\left( \theta \right)\), we
have
\begin{equation*}
    u\left( a,\: \theta \right) = f\left( \theta \right) = A_0 + \sum_{n = 1}^\infty \left[ A_n a^n \cos{\left( n \theta \right)} + B_n a^n \sin{\left( n \theta \right)} \right]
\end{equation*}
where \(A_0\), \(A_n a^n\), and \(B_n a^n\) are the Fourier coefficients of \(f\left( \theta \right)\).
\section{Nonhomogeneous Problems}
Nonhomogeneous PDEs are problems that involve a linear PDE of the form:
\begin{equation*}
    L u = F
\end{equation*}
possibly with a boundary condition of the form:
\begin{equation*}
    u\left( c,\: t \right) = a\left( t \right).
\end{equation*}
\subsection{Steady-State and Transient Solutions}
For \textbf{time-independent} non-homogeneities (i.e., when \(F =
F\left( x \right)\)), we can separate the solution of a nonhomogeneous
PDE into a \textbf{steady-state solution}, which is found by setting
the time derivative to zero, and a \textbf{transient solution}, which
will satisfy the homogeneous PDE and boundary conditions. The
steady-state solution takes the form,
\begin{equation*}
    u\left( x,\: t \right) = U\left( x \right),
\end{equation*}
where \(U\left( x \right)\) satisfies the PDE and boundary conditions, but not the initial condition for \(u\left( x,\: t \right)\).
To find the evolution of a system from an initial condition \(u\left(
x,\: 0 \right) = f\left( x \right)\), we need to find a transient
solution \(v\left( x,\: t \right)\):
\begin{equation*}
    v\left( x,\: t \right) = u\left( x,\: t \right) - U\left( x \right).
\end{equation*}
This solution satisfies the general solution
\begin{equation*}
    u\left( x,\: t \right) = v\left( x,\: t \right) + U\left( x \right).
\end{equation*}
After substitution, \(u\left( x,\: t \right)\) will be transformed into an homogeneous PDE that
can be solved using the methods described in the previous sections.
\subsection{Eigenfunction Expansion}
For a general nonhomogeneous term \(F = F\left( x,\: t \right)\), we
assume that the solution will take the form of an eigenfunction
expansion in one variable, where the eigenfunctions are those that come
from the homogeneous version of the problem, and the unknown
coefficients are functions of the other variable. Here the boundary
conditions must be homogeneous, and therefore, the PDE must be
transformed (i.e., via a subtraction), to cancel any nonhomogeneous
terms. This can be done by choosing an appropriate \(v\left( x,\: t
\right)\) which yields a new PDE.
\section{Integral Transforms}
\subsection{The Fourier Transform}
The Fourier series approximation is defined on the finite domain
\(\interval{-L}{L}\), where the approximation \(f_F\) is periodically
extended outside of this domain. The Fourier transform considers the
limiting process \(L \to \infty\) to obtain a transform defined on the
infinite domain \(\ointerval{-\infty}{\infty}\). Let \(\omega_n =
\frac{n \pi}{L}\) for \(n \in \Z\) so that \(\fdif{\omega} = \omega_{n
+ 1} - \omega_n = \frac{\pi}{L}\). Then
\begin{multline*}
    f_F\left( x \right) = \left[ \frac{1}{2\pi} \int_{-L}^L f\left( x \right) \odif{x} \right] \fdif{\omega}
    + \frac{1}{\pi} \sum_{n = 1}^\infty
    \left( \left[ \int_{-L}^L f\left( z \right) \cos{\left( \omega_n z \right)} \odif{z} \right] \cos{\left( \omega_n x \right)} \right. \\
    + \left. \left[ \int_{-L}^L f\left( z \right) \sin{\left( \omega_n z \right)} \odif{z} \right] \sin{\left( \omega_n x \right)} \right) \fdif{\omega}.
\end{multline*}
Taking the limit \(L \to \infty\) results in \(\fdif{\omega} \to 0\), so that
\begin{equation*}
    f_F\left( x \right) = \frac{1}{\pi} \int_0^\infty \left[ \int_{-\infty}^\infty f\left( z \right) \cos{\left( \omega z \right)} \odif{z} \right] \cos{\left( \omega x \right)} \odif{\omega} + \frac{1}{\pi} \int_0^\infty \left[ \int_{-\infty}^\infty f\left( z \right) \sin{\left( \omega z \right)} \odif{z} \right] \sin{\left( \omega x \right)} \odif{\omega}.
\end{equation*}
As the two integrands are even and odd respectively, we can use Euler's
identity to simplify the expression. First, let us define
\(g\left( \omega \right) = \int_{-\infty}^\infty f\left( z \right) \cos{\left( \omega z \right)} \odif{z}\)
and \(h\left( \omega \right) = \int_{-\infty}^\infty f\left( z \right) \sin{\left( \omega z \right)} \odif{z}\).
Then,
\begin{align*}
    f_F\left( x \right) & = \frac{1}{\pi} \int_0^\infty g\left( \omega \right) \cos{\left( \omega x \right)} \odif{\omega} + \frac{1}{\pi} \int_0^\infty h\left( \omega \right) \sin{\left( \omega x \right)} \odif{\omega}                                                                                                                    \\
                        & = \frac{1}{2 \pi} \int_0^\infty g\left( \omega \right) \left( e^{i \omega x} + e^{-i \omega x} \right) \odif{\omega} + \frac{1}{2i \pi} \int_0^\infty h\left( \omega \right) \left( e^{i \omega x} - e^{-i \omega x} \right) \odif{\omega}                                                                           \\
                        & = \frac{1}{2 \pi} \left[ \int_0^\infty g\left( \omega \right) e^{i \omega x} \odif{\omega} + \int_0^\infty g\left( \omega \right) e^{-i \omega x} \odif{\omega} - i \int_0^\infty h\left( \omega \right) e^{i \omega x} \odif{\omega} + i \int_0^\infty h\left( \omega \right) e^{-i \omega x} \odif{\omega} \right]
\end{align*}
Using the substitution \(u = -\omega\) for the second and fourth
integrals, the RHS becomes
\begin{equation*}
    \frac{1}{2 \pi} \left[ \int_0^\infty g\left( \omega \right) e^{i \omega x} \odif{\omega} - \int_0^{-\infty} g\left( -u \right) e^{i u x} \odif{u} - i \int_0^\infty h\left( \omega \right) e^{i \omega x} \odif{\omega} - i \int_0^{-\infty} h\left( -u \right) e^{i u x} \odif{u} \right].
\end{equation*}
Reverting this variable back to \(\omega\) allows for the following
simplification:
\begin{align*}
    f_F\left( x \right) & = \frac{1}{2 \pi} \left[ \int_0^\infty g\left( \omega \right) e^{i \omega x} \odif{\omega} + \int_{-\infty}^0 g\left( \omega \right) e^{i \omega x} \odif{\omega} - i \int_0^\infty h\left( \omega \right) e^{i \omega x} \odif{\omega} - i \int_{-\infty}^0 h\left( \omega \right) e^{i \omega x} \odif{\omega} \right] \\
                        & = \frac{1}{2 \pi} \left[ \int_{-\infty}^\infty g\left( \omega \right) e^{i \omega x} \odif{\omega} - i \int_{-\infty}^\infty h\left( \omega \right) e^{i \omega x} \odif{\omega} \right]                                                                                                                                 \\
                        & = \frac{1}{2 \pi} \int_{-\infty}^\infty \left[ \int_{-\infty}^\infty f\left( z \right) \left( \cos{\left( \omega z \right)} - i \sin{\left( \omega z \right)} \right) \odif{z} \right] e^{i \omega x} \odif{\omega}                                                                                                      \\
                        & = \frac{1}{2 \pi} \int_{-\infty}^\infty \left[ \int_{-\infty}^\infty f\left( z \right) e^{-i \omega z} \odif{z} \right] e^{i \omega x} \odif{\omega}.
\end{align*}
\begin{definition}[Fourier Transform]
    The Fourier transform of a function \(f\left( x \right)\) is denoted
    \(\hat{f}\left( \omega \right)\) and is defined:
    \begin{equation*}
        \hat{f}\left( \omega \right) = \mathscr{F}\left\{ f\left( x \right) \right\} = \int_{-\infty}^\infty f\left( x \right) e^{-i \omega x} \odif{x}.
    \end{equation*}
\end{definition}
\begin{definition}[Inverse Fourier Transform]
    The inverse Fourier transform of a function
    \(\hat{f}\left( \omega \right)\) is denoted
    \(f_F\left( x \right)\) and is defined:
    \begin{equation*}
        f_F\left( x \right) = \mathscr{F}^{-1}\left\{ f\left( x \right) \right\} = \frac{1}{2 \pi} \int_{-\infty}^\infty \hat{f}\left( \omega \right) e^{i \omega x} \odif{\omega}.
    \end{equation*}
\end{definition}
PDEs that are defined on infinite domains can utilise the Fourier
transform to transform the problem to an ODE in \(\omega\). In such
problems, it is required the solution be bounded as the spatial
variable approaches \(\pm \infty\).
\subsection{The Laplace Transform}
The Laplace transform is a generalisation of the Fourier transform that
is defined on the semi-infinite domain \(\ointerval{0}{\infty}\). The
\begin{definition}[Laplace Transform]
    The Laplace transform of a function \(f\left( t \right)\) is denoted
    \(\mathscr{L}\left\{ f\left( t \right) \right\}\) and is defined:
    \begin{equation*}
        F\left( s \right) = \mathscr{L}\left\{ f\left( t \right) \right\} = \int_0^\infty f\left( t \right) e^{-s t} \odif{t}.
    \end{equation*}
    where \(s\) is a complex variable. Note that for this integral to
    exist, \(f\) must not grow faster than \(e^{st}\) as \(t \to \infty\).
\end{definition}
PDEs that are defined on semi-infinite domains can utilise the Laplace
transform to transform the problem to an ODE of the remaining variable.
Once solved, the inverse Laplace transform can be used to find the
solution in the original domain.
\subsection{The Convolution}
The convolution of two functions \(f\left( x \right)\) and \(g\left( x
\right)\) is defined:
\begin{equation*}
    \left( f \ast g \right)\left( x \right) = \int_{-\infty}^\infty f\left( x - z \right) g\left( z \right) \odif{z}.
\end{equation*}
\begin{theorem}[Convolution Theorem for Fourier Transforms]
    \begin{equation*}
        \mathscr{F}\left\{ \left( f \ast g \right)\left( x \right) \right\} = \hat{f}\left( \omega \right) \hat{g}\left( \omega \right).
    \end{equation*}
    \begin{equation*}
        \frac{1}{2\pi} \left( \hat{f} \ast \hat{g} \right)\left( \omega \right) = \mathscr{F}\left\{ f\left( x \right) g\left( x \right) \right\}.
    \end{equation*}
\end{theorem}
\begin{theorem}[Convolution Theorem for Laplace Transforms]
    \begin{equation*}
        \mathscr{L}\left\{ \left( f \ast g \right)\left( t \right) \right\} = F\left( s \right) G\left( s \right).
    \end{equation*}
    \begin{equation*}
        f\left( t \right) g\left( t \right) = \mathscr{L}^{-1}\left\{ F\left( s \right) \ast G\left( s \right) \right\}.
    \end{equation*}
\end{theorem}
\section{Complex Analysis}
\subsection{Complex Numbers}
\subsubsection{Definition}
A complex number is any number of the form \(x + i y\) where \(x,y \in
\R\) and \(i^2 = -1\). Here, \(i\) is known as the imaginary unit. The
set of complex numbers \(\C\) can therefore be defined as
\begin{equation*}
    \C = \left\{ x + i y : x,y \in \R \right\}.
\end{equation*}
The real part of a complex number \(z = x + i y\) is denoted
\(\Re\left( z \right) = x\) and the imaginary part is denoted
\(\Im\left( z \right) = y\).
\subsubsection{Operations on Complex Numbers}
Addition and multiplication of two complex numbers is performed on the
real and imaginary components separately. All axioms for addition and
multiplication in \(\R\) are preserved in \(\C\):
\begin{align*}
    z_1 + z_2 & = \left( x_1 + i y_1 \right) + \left( x_2 + i y_2 \right) = \left( x_1 + x_2 \right) + i \left( y_1 + y_2 \right)                \\
    z_1 z_2   & = \left( x_1 + i y_1 \right) \left( x_2 + i y_2 \right) = \left( x_1 x_2 - y_1 y_2 \right) + i \left( x_1 y_2 + x_2 y_1 \right).
\end{align*}
\subsubsection{Polar Form}
A complex number \(z = x + i y\) can be expressed in polar form as
\begin{equation*}
    z = r \left( \cos{\left( \theta \right)} + i \sin{\left( \theta \right)} \right)
\end{equation*}
where \(r\) is the modulus of \(z\) defined
\begin{equation*}
    \abs*{z} = r = \sqrt{x^2 + y^2},
\end{equation*}
and \(\theta\) is the argument of \(z\) defined
\begin{equation*}
    \Arg\left( z \right) = \theta = \arctan{\left( \frac{y}{x} \right)}.
\end{equation*}
Euler's formula states that
\begin{equation*}
    e^{i \theta} = \cos{\left( \theta \right)} + i \sin{\left( \theta \right)},
\end{equation*}
this allows us to express the polar form of a complex number using the
exponential function:
\begin{equation*}
    z = r e^{i \theta}.
\end{equation*}
\subsubsection{Complex Conjugate}
The complex conjugate of a complex number \(z = x + i y\) is denoted
\(\overline{z} = x - i y\). This operation can be used to find the
modulus of a complex number:
\begin{equation*}
    \abs*{z} = \sqrt{z \overline{z}} \iff \abs*{z}^2 = z \overline{z}.
\end{equation*}
\subsubsection{Theorems on the Complex Plane}
\begin{theorem}[De Moivre's formula]
    For any complex number \(z = r e^{i \theta}\) and integer \(n\),
    \begin{equation*}
        z^n = r^n \left( \cos{\left( n \theta \right)} + i \sin{\left( n \theta \right)} \right).
    \end{equation*}
\end{theorem}
\begin{proof}
    Using Euler's formula, we have
    \begin{align*}
        z^n & = \left( r e^{i \theta} \right)^n = r^n e^{i n \theta} = r^n \left( \cos{\left( n \theta \right)} + i \sin{\left( n \theta \right)} \right) \\
            & = r^n \cos{\left( n \theta \right)} + i r^n \sin{\left( n \theta \right)}.
    \end{align*}
\end{proof}
\begin{theorem}[Triangle inequality for complex numbers]
    For any two complex numbers \(z_1\) and \(z_2\):
    \begin{equation*}
        \abs*{z_1 + z_2} \leqslant \abs*{z_1} + \abs*{z_2}.
    \end{equation*}
\end{theorem}
\begin{proof}
    We will prove this theorem by showing that this inequality is a
    tautology. Let \(z_1 = x_1 + i y_1\) and \(z_2 = x_2 + i y_2\).
    Then,
    \begin{align*}
        \abs*{z_1 + z_2}                                             & \leqslant \abs*{z_1} + \abs*{z_2}                                               \\
        \abs*{z_1 + z_2}^2                                           & \leqslant \left( \abs*{z_1} + \abs*{z_2} \right)^2                              \\
        \left( z_1 + z_2 \right) \overline{\left( z_1 + z_2 \right)} & \leqslant \abs*{z_1}^2 + 2 \abs*{z_1} \abs*{z_2} + \abs*{z_2}^2                 \\
        \left( x_1 + x_2 \right)^2 + \left( y_1 + y_2 \right)^2      & \leqslant x_1^2 + y_1^2 + 2 \abs*{z_1} \abs*{z_2} + x_2^2 + y_2^2               \\
        x_1^2 + 2 x_1 x_2 + x_2^2 + y_1^2 + 2 y_1 y_2 + y_2^2        & \leqslant x_1^2 + y_1^2 + 2 \abs*{z_1} \abs*{z_2} + x_2^2 + y_2^2               \\
        2 x_1 x_2 + 2 y_1 y_2                                        & \leqslant 2 \abs*{z_1} \abs*{z_2}                                               \\
        x_1 x_2 + y_1 y_2                                            & \leqslant \left( x_1^2 + y_1^2 \right)^{1/2} \left( x_2^2 + y_2^2 \right)^{1/2} \\
        \left( x_1 x_2 + y_1 y_2 \right)^2                           & \leqslant \left( x_1^2 + y_1^2 \right) \left( x_2^2 + y_2^2 \right)             \\
        x_1^2 x_2^2 + 2 x_1 x_2 y_1 y_2 + y_1^2 y_2^2                & \leqslant x_1^2 x_2^2 + x_1^2 y_2^2 + x_2^2 y_1^2 + y_1^2 y_2^2                 \\
        2 x_1 x_2 y_1 y_2                                            & \leqslant x_1^2 y_2^2 + x_2^2 y_1^2                                             \\
        0                                                            & \leqslant x_1^2 y_2^2 - 2 x_1 x_2 y_1 y_2 + x_2^2 y_1^2                         \\
        0                                                            & \leqslant \left( x_1 y_2 - x_2 y_1 \right)^2.
    \end{align*}
\end{proof}
\begin{theorem}[Reverse triangle inequality for complex numbers]
    For any two complex numbers \(z_1\) and \(z_2\):
    \begin{equation*}
        \abs*{\abs*{z_1} - \abs*{z_2}} \leqslant \abs*{z_1 - z_2}.
    \end{equation*}
\end{theorem}
\begin{proof}
    \begin{equation*}
        \abs*{z_1} = \abs*{\left( z_1 - z_2 \right) + z_2} \leqslant \abs*{z_1 - z_2} + \abs*{z_2} \implies \abs*{z_1} - \abs*{z_2} \leqslant \abs*{z_1 - z_2},
    \end{equation*}
    and
    \begin{equation*}
        \abs*{z_2} = \abs*{\left( z_2 - z_1 \right) + z_1} \leqslant \abs*{z_2 - z_1} + \abs*{z_1} \implies \abs*{z_2} - \abs*{z_1} \leqslant \abs*{z_2 - z_1}.
    \end{equation*}
    But
    \begin{align*}
        \abs*{z_1 - z_2} & = \sqrt{\left( x_1 - x_2 \right)^2 + \left( y_1 - y_2 \right)^2} \\
                         & = \sqrt{\left( x_2 - x_1 \right)^2 + \left( y_2 - y_1 \right)^2} \\
                         & = \abs*{z_2 - z_1}.
    \end{align*}
    Hence,
    \begin{equation*}
        \abs*{\abs*{z_1} - \abs*{z_2}} \leqslant \abs*{z_1 - z_2}.
    \end{equation*}
\end{proof}
\begin{corollary}[Triangle inequality for multiple complex numbers]
    The triangle inequality can be extended to any number of complex
    numbers:
    \begin{equation*}
        \abs*{\sum_{i = 1}^n z_i} \leqslant \sum_{i = 1}^n \abs*{z_i}.
    \end{equation*}
\end{corollary}
\begin{proof}
    We will prove this corollary by induction. The base case with
    \(n = 1\) complex numbers is trivially true:
    \begin{equation*}
        \abs*{z_1} \leqslant \abs*{z_1}.
    \end{equation*}
    If we assume that this corollary holds for \(n = k\) complex
    numbers:
    \begin{equation*}
        \abs*{\sum_{i = 1}^k z_i} \leqslant \sum_{i = 1}^k \abs*{z_i}.
    \end{equation*}
    we can show that this corollary also holds for \(n = k + 1\) complex
    numbers:
    \begin{equation*}
        \abs*{\sum_{i = 1}^{k + 1} z_i} = \abs*{\sum_{i = 1}^k z_i + z_{k + 1}} \leqslant \abs*{\sum_{i = 1}^k z_i} + \abs*{z_{k + 1}} \leqslant \sum_{i = 1}^k \abs*{z_i} + \abs*{z_{k + 1}} = \sum_{i = 1}^{k + 1} \abs*{z_i}.
    \end{equation*}
    Since this corollary holds for the base case, and also for the
    inductive step whenever the inductive hypothesis is true, the
    corollary holds for all \(n \in \N\).
\end{proof}
\subsection{Complex-Valued Functions}
A complex-valued function \(f\) maps the complex values \(z\) in a set
\(S\) to a unique set of complex values \(w\) in a set \(T\). The
function \(f\) is defined:
\begin{equation*}
    w = f\left( z \right).
\end{equation*}
Separating the real and imaginary components of \(z\) and \(w\) as
\(z = x + i y\) and \(w = u + i v\), shows that \(u\) and \(v\) are
functions of \(x\) and \(y\):
\begin{equation*}
    w = f\left( z \right) = u\left( x,\: y \right) + i v\left( x,\: y \right).
\end{equation*}
Therefore \(f\) is a mapping of values from the
\(\left( x,\: y \right)\) plane to the \(\left( u,\: v \right)\) plane.
\subsubsection{Complex Exponentials}
The complex exponential is defined
\begin{equation*}
    w = e^z = \exp{\left( z \right)} = e^x \left( \cos{\left( y \right)} + i \sin{\left( y \right)} \right).
\end{equation*}
Using this form, it can be shown that the complex exponential maintains
several properties of the real exponential:
\begin{enumerate}
    \item \(e^{z_1 + z_2} = e^{z_1} e^{z_2}\)
    \item \(e^{n z} = \left( e^z \right)^n\) for \(n \in \Z\)
\end{enumerate}
\subsubsection{Trigonometric and Hyperbolic Functions}
Using Euler's formula, we can derive the exponential form of the
following trigonometric functions:
\begin{align*}
    \cos{\left( z \right)} = \frac{e^{i z} + e^{-i z}}{2} \qquad \sin{\left( z \right)} = \frac{e^{i z} - e^{-i z}}{2i} \\
    \cosh{\left( z \right)} = \frac{e^{z} + e^{-z}}{2} \qquad \sinh{\left( z \right)} = \frac{e^{z} - e^{-z}}{2}
\end{align*}
from this, it can be deduced that
\begin{equation*}
    \cos{\left( z \right)} = \cosh{\left( i z \right)} \quad \text{and} \quad i \sin{\left( z \right)} = \sinh{\left( i z \right)}
\end{equation*}
or similarly,
\begin{equation*}
    \cosh{\left( z \right)} = \cos{\left( i z \right)} \quad \text{and} \quad i \sinh{\left( z \right)} = \sin{\left( i z \right)}.
\end{equation*}
\subsubsection{Complex Logarithms}
The natural logarithm of a complex number \(z\), is denoted
\begin{equation*}
    w = \log{\left( z \right)}
\end{equation*}
and is defined as the solution to \(e^w = z\) for all \(z \neq 0\).
Using polar form, we can use the periodicity of the complex exponential
to show that there are infinitely many solutions to this equation.
\begin{align*}
    \log{\left( z \right)} & = \ln{\left( \abs*{z} e^{i \arg{\left( z \right)}} \right)}                        \\
                           & = \ln{\left( \abs*{z} e^{i \left( \Arg{\left( z \right)} + 2\pi n\right)} \right)} \\
                           & = \ln{\abs*{z}} + i \left( \Arg{\left( z \right)} + 2 \pi n \right),
\end{align*}
where \(\arg{\left( z \right)}\) is the argument of \(z\), defined using
the principal value of the argument, \(\Arg{\left( z \right)}\), and
\(n \in \Z\). Here the function \(\ln\) corresponds to the real-valued
natural logarithm. If we wish to define a principal value \(\Log{\left( z
    \right)}\) whose imaginary part lies in the interval
\(\linterval{-\pi}{\pi}\), we can use the principal value of the
argument:
\begin{equation*}
    \Log{\left( z \right)} = \ln{\abs*{z}} + i \Arg{\left( z \right)}.
\end{equation*}
This lets us also define the complex logarithm in terms of the principal
logarithm:
\begin{equation*}
    \log{\left( z \right)} = \Log{\left( z \right)} + 2 \pi i n.
\end{equation*}
\subsubsection{Analytic Functions}
In this section, we discuss the idea of differentiability for complex
functions. Let \(w = f\left( z \right)\) be defined in a neighbourhood
\(0 < \abs*{z - z_0} < \delta\) of \(z_0\), except possibly at some
isolated points. We say
\begin{equation*}
    \lim_{z \to z_0} f\left( z \right) = w_0
\end{equation*}
if for any \(\epsilon > 0\), there exists a \(\delta > 0\) such that
\begin{equation*}
    \abs*{f\left( z \right) - w_0} < \epsilon
\end{equation*}
for all \(0 < \abs*{z - z_0} < \delta\)\footnote{Note that it is not
    necessary for the function to be defined at \(z_0\) itself: \(f\left( z_0 \right)
    = w_0\).}. As with real functions, this limit must be independent of the
path taken to approach \(z_0\). The derivative of a complex function
\(f\left( z \right)\) is defined as
\begin{equation*}
    \odv{f}{z} = \lim_{\adif{z} \to 0} \frac{f\left( z + \adif{z} \right) - f\left( z \right)}{\adif{z}},
\end{equation*}
provided this limit exists. It follows that all rules of differentiation
for real functions apply to complex functions. Recall that we can
express a complex function in terms of its real and imaginary parts:
\begin{equation*}
    f\left( z \right) = u\left( x,\: y \right) + i v\left( x,\: y \right).
\end{equation*}
Therefore, let us decompose the derivative of \(f\) into its real and
imaginary parts, using \(\adif{z} = \adif{x} + i \adif{y}\):
\begin{equation*}
    \odv{f}{z} = \lim_{\substack{\adif{x} \to 0 \\ \adif{y} \to 0}} \frac{u\left( x + \adif{x},\: y + \adif{y} \right) + i v\left( x + \adif{x},\: y + \adif{y} \right) - u\left( x,\: y \right) - i v\left( x,\: y \right)}{\adif{x} + i \adif{y}}.
\end{equation*}
For this limit to exist, the value must be independent of which variable
approaches zero first. If we consider the limit as \(\adif{y} \to 0\)
first, we find
\begin{align*}
    \odv{f}{z} & = \lim_{\adif{x} \to 0} \frac{u\left( x + \adif{x},\: y \right) - u\left( x,\: y \right)}{\adif{x}} + i \lim_{\adif{x} \to 0} \frac{v\left( x + \adif{x},\: y \right) - v\left( x,\: y \right)}{\adif{x}} \\
               & = \pdv{u}{x} + i \pdv{v}{x}.
\end{align*}
Similarly, if we consider the limit as \(\adif{x} \to 0\) first, we find
\begin{align*}
    \odv{f}{z} & = \lim_{\adif{y} \to 0} \frac{u\left( x,\: y + \adif{y} \right) - u\left( x,\: y \right)}{i \adif{y}} + i \lim_{\adif{y} \to 0} \frac{v\left( x,\: y + \adif{y} \right) - v\left( x,\: y \right)}{i \adif{y}} \\
               & = \frac{1}{i} \pdv{u}{y} + \pdv{v}{y}                                                                                                                                                                         \\
               & = \pdv{v}{y} - i \pdv{u}{y}.
\end{align*}
Therefore, the complex derivative of a function is given by:
\begin{equation*}
    \odv{f}{z} = \pdv{u}{x} + i \pdv{v}{x} = \pdv{v}{y} - i \pdv{u}{y}.
\end{equation*}
This implies that the following relationships hold between \(u\) and
\(v\):
\begin{equation*}
    \pdv{u}{x} = \pdv{v}{y} \quad \text{and} \quad \pdv{u}{y} = -\pdv{v}{x}.
\end{equation*}
These equations are known as the \textbf{Cauchy-Riemann equations}, and
are necessary conditions for the complex derivative to exist at a point,
provided that the partial derivatives of \(u\) and \(v\) are continuous.
A function that is complex differentiable in the \textbf{neighbourhood}
of a point is said to be \textbf{analytic} at that point. Such functions are
infinitely smooth (differentiable) and have a convergent Taylor series
expansion about that point. A function that is analytic on the entire
complex plane is said to be \textbf{entire}.
\subsubsection{Common Analytic Functions}
Using the definition of the complex derivative, we can show that the
following functions are analytic in the complex plane:
\begin{itemize}
    \item Power functions \(z^n\) for \(n \in \Z\)
    \item Exponential functions \(e^z\)
    \item Trigonometric functions \(\cos{\left( z \right)}\) and
          \(\sin{\left( z \right)}\)
    \item Hyperbolic functions \(\cosh{\left( z \right)}\) and
          \(\sinh{\left( z \right)}\)
    \item Rational functions \(f\left( z \right) = \frac{P\left( z
          \right)}{Q\left( z \right)}\) (except when \(Q\left( z
          \right) = 0\))
    \item Any linear combination or composition of the above functions
    \item Logarithmic, non-integer power, and inverse trigonometric
          functions (except at branch points or branch cuts)
\end{itemize}
Additionally, all rules of differentiation for real functions also apply
to complex functions.
\subsubsection{Harmonic Functions}
The Cauchy-Riemann equations allow us to relate a complex function to
Laplace's equation. Consider a complex function \(f\left( z \right) =
u\left( x,\: y \right) + i v\left( x,\: y \right)\) that is analytic in
a domain \(\mathcal{D}\). Then, we can show that both \(u\) and \(v\)
satisfy Laplace's equation using the Cauchy-Riemann equations:
\begin{align*}
    \pdv{u}{x}          & = \pdv{v}{y},          & \pdv{u}{y}          & = -\pdv{v}{x}                                                                                           \\
    \pdv[order=2]{u}{x} & = \pdv{v}{x,y},        & \pdv[order=2]{u}{y} & = -\pdv{v}{y,x}        & \implies \pdv[order=2]{u}{x} = -\pdv[order=2]{u}{y} & \implies \nabla^2 u = 0. \\
    \pdv{u}{y,x}        & = \pdv[order=2]{v}{y}, & \pdv{u}{x,y}        & = -\pdv[order=2]{v}{x} & \implies \pdv[order=2]{v}{y} = -\pdv[order=2]{v}{x} & \implies \nabla^2 v = 0.
\end{align*}
Here we use Clairaut's theorem to show that the mixed partial derivatives
are equal. If \(u\) and \(v\) satisfy Laplace's equations and the
Cauchy-Riemann equations, they are said to be \textbf{harmonic
    conjugates}.
\subsection{Contour Integrals}
Consider the oriented curve \(\mathcal{C}\) in the complex plane, which
is parameterised by \(z = z\left( t \right)\) for \(a \leqslant t
\leqslant b\):
\begin{equation*}
    z\left( t \right) = x\left( t \right) + i y\left( t \right),
\end{equation*}
where \(x\) and \(y\) are real continuous functions. A curve is said to
be:
\begin{itemize}
    \item (piecewise) \textbf{smooth} if \(\odv{z}{t} = \odv{x}{t} + i \odv{y}{t}\) is (piecewise) continuous and nonzero for all \(t\)
    \item \textbf{closed} if \(z\left( a \right) = z\left( b \right)\)
    \item \textbf{simple} if it does not intersect itself: \(z\left( t_1 \right) \neq z\left( t_2 \right)\) for \(t_1 \neq t_2\)
\end{itemize}
The line integral of a complex function \(f\left( z \right)\) along a
curve \(\mathcal{C}\) is defined as a contour integral:
\begin{equation*}
    \int_{\mathcal{C}} f\left( z \right) \odif{z} = \int_a^b f\left( z\left( t \right) \right) \odv{z}{t} \odif{t}.
\end{equation*}
A useful result in complex analysis is the closed line integral of the
complex power function:
\begin{equation*}
    \oint_{\mathcal{C}_R} z^n \odif{z}
\end{equation*}
where \(\mathcal{C}_R\) is a circle of radius \(R\) oriented anticlockwise.
We can solve this integral using the parametrisation \(z\left( t \right) = R e^{i t}\)
for \(0 \leqslant t \leqslant 2 \pi\):
\begin{align*}
    \oint_{\mathcal{C}_R} z^n \odif{z} & = \int_0^{2 \pi} \left( R e^{i t} \right)^n i R e^{i t} \odif{t}                                    \\
                                       & = i R^{n + 1} \int_0^{2 \pi} e^{i \left( n + 1 \right) t} \odif{t}                                  \\
                                       & = i R^{n + 1} \left[ \frac{e^{i \left( n + 1 \right) t}}{i \left( n + 1 \right)} \right]_0^{2 \pi}.
\end{align*}
When \(n \neq -1\), the integral evaluates to zero, as the exponential
function is periodic with period \(2 \pi\). When \(n = -1\), the second
line evaluates to:
\begin{equation*}
    i R^0 \int_0^{2 \pi} \odif{t} = 2 \pi i.
\end{equation*}
Therefore,
\begin{equation*}
    \oint_{\mathcal{C}_R} z^n \odif{z} =
    \begin{cases}
        2 \pi i & n = -1,    \\
        0       & n \neq -1.
    \end{cases}
\end{equation*}
\subsubsection{Contour Integral Theorems}
\begin{theorem}[Cauchy's integral theorem]
    Let \(f\left( z \right)\) be analytic on and within a simple closed
    curve \(\mathcal{C}\):
    \begin{equation*}
        \oint_{\mathcal{C}} f\left( z \right) \odif{z} = 0.
    \end{equation*}
\end{theorem}
\begin{proof}
    Let \(f\left( z \right) = u\left( x,\: y \right) + i v\left( x,\: y \right)\)
    be analytic on and within \(\mathcal{C}\). Then,
    \begin{align*}
        \oint_{\mathcal{C}} f\left( z \right) \odif{z} & = \oint_{\mathcal{C}} \left( u\left( x,\: y \right) + i v\left( x,\: y \right) \right) \left( \odif{x} + i \odif{y} \right)                                                                                      \\
                                                       & = \oint_{\mathcal{C}} \left( u\left( x,\: y \right) \odif{x} - v\left( x,\: y \right) \odif{y} \right) + i \oint_{\mathcal{C}} \left( v\left( x,\: y \right) \odif{x} + u\left( x,\: y \right) \odif{y} \right).
    \end{align*}
    By Green's theorem, this becomes,
    \begin{equation*}
        \oint_{\mathcal{C}} f\left( z \right) \odif{z} = \iint_{\mathcal{D}} \left( \pdv{v}{x} - \pdv{u}{y} \right) \odif{x} \odif{y} + i \iint_{\mathcal{D}} \left( \pdv{u}{x} + \pdv{v}{y} \right) \odif{x} \odif{y},
    \end{equation*}
    where \(\mathcal{D}\) is the region enclosed by \(\mathcal{C}\). As
    \(f\) is analytic, it satisfies the Cauchy-Riemann equations, so that
    both integrands are zero. Therefore, the integral is zero:
    \begin{equation*}
        \oint_{\mathcal{C}} f\left( z \right) \odif{z} = \iint_{\mathcal{D}} \left( 0 \right) \odif{x} \odif{y} + i \iint_{\mathcal{D}} \left( 0 \right) \odif{x} \odif{y} = 0.
    \end{equation*}
\end{proof}
\begin{corollary}[Path independence]
    The contour integral between any two points \(z_1\) and \(z_2\) is
    independent of path, provided that the path is in a region where the
    integrand is analytic.
\end{corollary}
\begin{proof}
    Let \(\mathcal{C}_1\) and \(\mathcal{C}_2\) be two curves connecting
    the points \(z_1\) and \(z_2\) in a region where the function
    \(f\left( z \right)\) is analytic. Furthermore, the closed curve
    \(\mathcal{C} = \mathcal{C}_1 - \mathcal{C}_2\) is a simple closed
    curve which encloses no singularities. By Cauchy's integral theorem,
    \begin{align*}
        \oint_{\mathcal{C}} f\left( z \right) \odif{z}                                                      & = 0                                                 \\
        \oint_{\mathcal{C}_1} f\left( z \right) \odif{z} - \oint_{\mathcal{C}_2} f\left( z \right) \odif{z} & = 0                                                 \\
        \oint_{\mathcal{C}_1} f\left( z \right) \odif{z}                                                    & = \oint_{\mathcal{C}_2} f\left( z \right) \odif{z}.
    \end{align*}
\end{proof}
\begin{corollary}[Equivalence of homotopic contours]
    The contour integral over any two closed contours in a region where
    the integrand is analytic are equal, provided that the contours can
    be continuously deformed into each other without crossing any
    singularities. Such contours are said to be \textbf{homotopic}.
\end{corollary}
\begin{proof}
    Let \(\mathcal{C}_1\) and \(\mathcal{C}_2\) be two closed contours
    defined in a region where the function \(f\left( z \right)\) is
    analytic and assume \(\mathcal{C}_1\) is enclosed by
    \(\mathcal{C}_2\). Then, we can define a closed curve
    \(\mathcal{C} = \mathcal{C}_1 + \mathcal{C}_3 - \mathcal{C}_2 - \mathcal{C}_3\),
    where \(\mathcal{C}_3\) is a curve that connects the endpoints of
    \(\mathcal{C}_1\) and \(\mathcal{C}_2\) in both directions. By
    Cauchy's integral theorem,
    \begin{equation*}
        \oint_{\mathcal{C}} f\left( z \right) \odif{z} = 0.
    \end{equation*}
    The contribution of \(\mathcal{C}_3\) and \(-\mathcal{C}_3\) cancel
    out, which leaves
    \begin{align*}
        \int_{\mathcal{C}_1 - \mathcal{C}_2} f\left( z \right) \odif{z} & = 0                                                \\
        \int_{\mathcal{C}_1} f\left( z \right) \odif{z}                 & = \int_{\mathcal{C}_2} f\left( z \right) \odif{z}.
    \end{align*}
\end{proof}
\begin{theorem}[Cauchy's integral formula for functions]
    Let \(f\left( z \right)\) be analytic on and within a simple closed
    curve \(\mathcal{C}\), and let \(z_0\) be a point within
    \(\mathcal{C}\). Then,
    \begin{equation*}
        f\left( z_0 \right) = \frac{1}{2 \pi i} \oint_{\mathcal{C}} \frac{f\left( z \right)}{z - z_0} \odif{z}.
    \end{equation*}
    Note we may alternatively write this as
    \begin{equation*}
        f\left( z \right) = \frac{1}{2 \pi i} \oint_{\mathcal{C}} \frac{f\left( w \right)}{w - z} \odif{w}
    \end{equation*}
    for an arbitrary point \(z\) within \(\mathcal{C}\).
\end{theorem}
\begin{proof}
    The function
    \begin{equation*}
        \frac{f\left( z \right)}{z - z_0}
    \end{equation*}
    is analytic on and within \(\mathcal{C}\), except at \(z = z_0\).
    By Cauchy's integral theorem, we can deform \(\mathcal{C}\) into a
    circle \(\mathcal{C}_R\) of radius \(R\) centred at \(z_0\),
    oriented in the same direction as \(\mathcal{C}\):
    \begin{equation*}
        \oint_{\mathcal{C}} \frac{f\left( z \right)}{z - z_0} \odif{z} = \oint_{\mathcal{C}_R} \frac{f\left( z \right)}{z - z_0} \odif{z}.
    \end{equation*}
    As \(f\) is analytic, it must be continuous, and \(f\left( z_0 \right)\)
    must be well-defined. Therefore, by taking the limit \(R \to 0\), we
    have concluded our proof:
    \begin{equation*}
        \oint_{\mathcal{C}_R} \frac{f\left( z \right)}{z - z_0} \odif{z} = \lim_{R \to 0} \oint_{\mathcal{C}_R} \frac{f\left( z \right)}{z - z_0} \odif{z} = \oint_{\mathcal{C}_R} \frac{f\left( z_0 \right)}{z - z_0} \odif{z} = f\left( z_0 \right) \oint_{\mathcal{C}_R} \frac{1}{z - z_0} \odif{z} = f\left( z_0 \right) 2 \pi i.
    \end{equation*}
\end{proof}
\begin{theorem}[Cauchy's integral formula for derivatives]
    Let \(f\left( z \right)\) be analytic on and within a simple closed
    curve \(\mathcal{C}\), then,
    \begin{equation*}
        f^{\left( n \right)}\left( z \right) = \frac{n!}{2 \pi i} \oint_{\mathcal{C}} \frac{f\left( w \right)}{\left( w - z \right)^{n + 1}} \odif{w}.
    \end{equation*}
\end{theorem}
\begin{proof}
    We will prove this result using induction. For the base case, we
    will consider the limit definition of the first derivative of \(f\):
    \begin{equation*}
        f'\left( z \right) = \lim_{\adif{z} \to 0} \frac{f\left( z + \adif{z} \right) - f\left( z \right)}{\adif{z}}.
    \end{equation*}
    We can express both terms in the numerator using the Cauchy integral
    formula:
    \begin{equation*}
        f\left( z + \adif{z} \right) = \frac{1}{2 \pi i} \oint_{\mathcal{C}} \frac{f\left( w \right)}{w - z - \adif{z}} \odif{w}, \qquad f\left( z \right) = \frac{1}{2 \pi i} \oint_{\mathcal{C}} \frac{f\left( w \right)}{w - z} \odif{w},
    \end{equation*}
    so that
    \begin{align*}
        f'\left( z \right) & = \lim_{\adif{z} \to 0} \frac{1}{\adif{z}} \left[ \frac{1}{2 \pi i} \oint_{\mathcal{C}} \frac{f\left( w \right)}{w - z - \adif{z}} \odif{w} - \frac{1}{2 \pi i} \oint_{\mathcal{C}} \frac{f\left( w \right)}{w - z} \odif{w} \right] \\
                           & = \lim_{\adif{z} \to 0} \frac{1}{2 \pi i \adif{z}} \oint_{\mathcal{C}} f\left( w \right) \left[ \frac{1}{w - z - \adif{z}} - \frac{1}{w - z} \right] \odif{w}                                                                        \\
                           & = \lim_{\adif{z} \to 0} \frac{1}{2 \pi i \adif{z}} \oint_{\mathcal{C}} \frac{f\left( w \right) \adif{z}}{\left( w - z - \adif{z} \right) \left( w - z \right)} \odif{w}                                                              \\
                           & = \lim_{\adif{z} \to 0} \frac{1}{2 \pi i} \oint_{\mathcal{C}} \frac{f\left( w \right)}{\left( w - z \right)^2 - \adif{z} \left( w - z \right)} \odif{w}                                                                              \\
                           & = \frac{1}{2 \pi i} \oint_{\mathcal{C}} \frac{f\left( w \right)}{\left( w - z \right)^2} \odif{w}.
    \end{align*}
    We will now assume that this theorem holds for \(n = k\),
    \begin{equation*}
        f^{\left( k \right)}\left( z \right) = \frac{k!}{2 \pi i} \oint_{\mathcal{C}} \frac{f\left( w \right)}{\left( w - z \right)^{k + 1}} \odif{w},
    \end{equation*}
    and show that it also holds when \(n = k + 1\).
    We will again use the limit definition of the derivative for this
    step:
    \begin{equation*}
        f^{\left( k + 1 \right)}\left( z \right) = \lim_{\adif{z} \to 0} \frac{f^{\left( k \right)}\left( z + \adif{z} \right) - f^{\left( k \right)}\left( z \right)}{\adif{z}}.
    \end{equation*}
    Using the induction hypothesis, we can express the terms in the
    numerator as:
    \begin{equation*}
        f^{\left( k \right)}\left( z + \adif{z} \right) = \frac{k!}{2 \pi i} \oint_{\mathcal{C}} \frac{f\left( w \right)}{\left( w - z - \adif{z} \right)^{k + 1}} \odif{w}, \qquad f^{\left( k \right)}\left( z \right) = \frac{k!}{2 \pi i} \oint_{\mathcal{C}} \frac{f\left( w \right)}{\left( w - z \right)^{k + 1}} \odif{w}.
    \end{equation*}
    Then,
    \begin{align*}
        f^{\left( k + 1 \right)}\left( z \right) & = \lim_{\adif{z} \to 0} \frac{1}{\adif{z}} \left[ \frac{k!}{2 \pi i} \oint_{\mathcal{C}} \frac{f\left( w \right)}{\left( w - z - \adif{z} \right)^{k + 1}} \odif{w} - \frac{k!}{2 \pi i} \oint_{\mathcal{C}} \frac{f\left( w \right)}{\left( w - z \right)^{k + 1}} \odif{w} \right] \\
                                                 & = \lim_{\adif{z} \to 0} \frac{k!}{2 \pi i \adif{z}} \oint_{\mathcal{C}} f\left( w \right) \left[ \frac{1}{\left( w - z - \adif{z} \right)^{k + 1}} - \frac{1}{\left( w - z \right)^{k + 1}} \right] \odif{w}                                                                         \\
                                                 & = \lim_{\adif{z} \to 0} \frac{k!}{2 \pi i \adif{z}} \oint_{\mathcal{C}} f\left( w \right) \frac{\left( w - z \right)^{k + 1} - \left( w - z - \adif{z} \right)^{k + 1}}{\left( w - z - \adif{z} \right)^{k + 1} \left( w - z \right)^{k + 1}} \odif{w}                               \\
                                                 & = \lim_{\adif{z} \to 0} \frac{k!}{2 \pi i \adif{z}} \oint_{\mathcal{C}} f\left( w \right) \frac{\left( w - z \right)^{k + 1} - \left( w - z - \adif{z} \right)^{k + 1}}{\left( \left( w - z \right)^2 - \adif{z} \left( w - z \right) \right)^{k + 1}} \odif{w}.
    \end{align*}
    To simplify the numerator of the integrand, we will use difference
    of \(n\)th powers formula:
    \begin{equation*}
        a^n - b^n = \left( a - b \right) \sum_{j = 0}^{n - 1} a^{n - 1 - j} b^j,
    \end{equation*}
    which greatly simplifies the numerator:
    \begin{align*}
        \left( w - z \right)^{k + 1} - \left( w - z - \adif{z} \right)^{k + 1} & = \left( w - z - w + z + \adif{z} \right) \sum_{j = 0}^k \left( w - z \right)^{k - j} \left( w - z - \adif{z} \right)^j                                       \\
                                                                               & = \adif{z} \sum_{j = 0}^k \left( w - z \right)^{k - j} \left( w - z - \adif{z} \right)^j                                                                      \\
                                                                               & = \adif{z} \left[ \left( w - z \right)^k + \left( w - z \right)^{k - 1} \left( w - z - \adif{z} \right) + \cdots + \left( w - z - \adif{z} \right)^k \right].
    \end{align*}
    Let us then substitute this result into the integrand:
    \begin{align*}
        f^{\left( k + 1 \right)}\left( z \right) & = \lim_{\adif{z} \to 0} \frac{k!}{2 \pi i \adif{z}} \oint_{\mathcal{C}} \frac{f\left( w \right) \adif{z} \left[ \left( w - z \right)^k + \cdots + \left( w - z - \adif{z} \right)^k \right]}{\left( \left( w - z \right)^2 - \adif{z} \left( w - z \right) \right)^{k + 1}} \odif{w} \\
                                                 & = \frac{k!}{2 \pi i} \oint_{\mathcal{C}} \frac{f\left( w \right) \left[ \left( w - z \right)^k + \left( w - z \right)^{k - 1} \left( w - z \right) + \cdots + \left( w - z \right)^k \right]}{\left( \left( w - z \right)^2 \right)^{k + 1}} \odif{w}                                \\
                                                 & = \frac{k!}{2 \pi i} \oint_{\mathcal{C}} \frac{f\left( w \right) \left( k + 1 \right) \left( w - z \right)^k}{\left( w - z \right)^{2k + 2}} \odif{w}                                                                                                                                \\
                                                 & = \frac{\left( k + 1 \right)!}{2 \pi i} \oint_{\mathcal{C}} \frac{f\left( w \right)}{\left( w - z \right)^{\left( k + 1 \right) + 1}} \odif{w}.
    \end{align*}
    Since the theorem holds for the base case, and also for the
    inductive step whenever the inductive hypothesis is true, the
    theorem holds for all \(n \in \mathbb{N}\).
\end{proof}
\begin{corollary}[Differentiability of analytic functions]
    If \(f\left( z \right)\) is analytic on and within a simple closed
    curve \(\mathcal{C}\), then \(f\) is infinitely differentiable on
    \(\mathcal{C}\).
\end{corollary}
\begin{proof}
    Cauchy's integral formula for derivatives provides an explicit
    expression for \(f^{\left( n \right)}\left( z \right)\) in terms
    of a contour integral, for all \(n \in \mathbb{N}\).
\end{proof}
\begin{theorem}[Triangle inequality for contour integrals]
    Let \(f\left( z \right)\) be analytic on and within a simple closed
    curve \(\mathcal{C}\), parametrised by \(z = z\left( t \right)\) for
    \(a \leqslant t \leqslant b\). Then,
    \begin{equation*}
        \abs*{\int_{\mathcal{C}} f\left( z \right) \odif{z}} \leqslant \int_{\mathcal{C}} \abs*{f\left( z \right)} \abs*{\odif{z}},
    \end{equation*}
    where \(\odif{z} = z'\left( t \right) \odif{t}\) and
    \(\abs*{\odif{z}} = \abs*{z'\left( t \right)} \odif{t}\).
\end{theorem}
\begin{proof}
    Consider the Riemann sum of the parametrised integral:
    \begin{equation*}
        \sum_{i = 1}^n f\left( z\left( t_i \right) \right) z'\left( t_i \right) \adif{t},
    \end{equation*}
    where \(\adif{t} = \frac{b - a}{n}\). By the triangle inequality for
    \(n\) complex numbers, we have
    \begin{equation*}
        \abs*{\sum_{i = 1}^n f\left( z\left( t_i \right) \right) z'\left( t_i \right) \adif{t}} \leqslant \sum_{i = 1}^n \abs*{f\left( z\left( t_i \right) \right) z'\left( t_i \right) \adif{t}} = \sum_{i = 1}^n \abs*{f\left( z\left( t_i \right) \right)} \abs*{z'\left( t_i \right)} \adif{t}.
    \end{equation*}
    If we consider the limit as \(n \to \infty\), we find
    \begin{equation*}
        \abs*{\int_a^b f\left( z\left( t \right) \right) z'\left( t \right) \odif{t}} \leqslant \int_a^b \abs*{f\left( z\left( t \right) \right)} \abs*{z'\left( t \right)} \odif{t} \iff \abs*{\int_{\mathcal{C}} f\left( z \right) \odif{z}} \leqslant \int_{\mathcal{C}} \abs*{f\left( z \right)} \abs*{\odif{z}}.
    \end{equation*}
\end{proof}
\begin{corollary}[Estimation of contour integrals]
    If \(\abs*{f\left( z \right)} \leqslant M\) for all \(z\) on and
    within a simple closed curve \(\mathcal{C}\) parametrised by
    \(z = z\left( t \right)\) for \(a \leqslant t \leqslant b\), then
    \begin{equation*}
        \abs*{\int_{\mathcal{C}} f\left( z \right) \odif{z}} \leqslant M \int_{\mathcal{C}} \abs*{\odif{z}} = M L,
    \end{equation*}
    where \(L\) is the length of the path \(\mathcal{C}\):
    \begin{equation*}
        L = \int_a^b \abs*{z'\left( t \right)} \odif{t}.
    \end{equation*}
\end{corollary}
\begin{proof}
    By the triangle inequality for contour integrals, we have
    \begin{equation*}
        \abs*{\int_{\mathcal{C}} f\left( z \right) \odif{z}} \leqslant \int_{\mathcal{C}} \abs*{f\left( z \right)} \abs*{\odif{z}} \leqslant M \int_{\mathcal{C}} \abs*{\odif{z}} = M L.
    \end{equation*}
\end{proof}
\subsection{Complex Series}
\begin{theorem}[Cauchy's root test]
    Given a series of complex numbers
    \begin{equation*}
        \sum_{n = 0}^\infty a_n,
    \end{equation*}
    consider the limit superior
    \begin{equation*}
        C = \limsup_{n \to \infty} \abs*{a_n}^{1/n}.
    \end{equation*}
    If
    \begin{itemize}
        \item \(C < 1\): the series converges absolutely, and
        \item \(C > 1\): the series diverges.
    \end{itemize}
    Note that this test is inconclusive when \(C = 1\).
\end{theorem}
\begin{proof}
    We will consider each case separately.
    When \(C < 1\),
    \begin{equation*}
        \limsup_{n \to \infty} \abs*{a_n}^{1/n} < 1.
    \end{equation*}
    Now consider some \(s \in \R\), where
    \begin{equation*}
        \limsup_{n \to \infty} \abs*{a_n}^{1/n} < s < 1.
    \end{equation*}
    As a consequence of the definition of the limit superior,
    \(\abs*{a_n}^{1/n}\) will be less than \(s\) for most values of
    \(n\):
    \begin{equation*}
        \abs*{a_n}^{1/n} < s \implies \abs*{a_n} < s^n.
    \end{equation*}
    Note that the series \(\sum_{n = 0}^\infty s^n\)
    converges by the geometric series test as \(s < 1\). Therefore, by
    the comparison test, the series \(\sum_{n = 0}^\infty \abs*{a_n}\)
    also converges. Thus, the series \(\sum_{n = 0}^\infty a_n\)
    converges absolutely. When \(C > 1\), we can use a similar argument
    and choose some \(s \in \R\) such that
    \begin{equation*}
        \limsup_{n \to \infty} \abs*{a_n}^{1/n} > s > 1.
    \end{equation*}
    Here, we can show that \(\abs*{a_n}^{1/n}\) will be greater than
    \(s\) for most values of \(n\):
    \begin{equation*}
        \abs*{a_n}^{1/n} > s \implies \abs*{a_n} > s^n.
    \end{equation*}
    As \(\sum_{n = 0}^\infty s^n\) diverges the series
    \(\sum_{n = 0}^\infty a_n\) also diverges.
\end{proof}
\begin{theorem}[Cauchy-Hadamard theorem]
    Given a complex power series
    \begin{equation*}
        \sum_{n = 0}^\infty a_n \left( z - z_0 \right)^n,
    \end{equation*}
    consider the limit superior
    \begin{equation*}
        \frac{1}{R} = \limsup_{n \to \infty} \abs*{a_n}^{1/n},
    \end{equation*}
    where we say \(R = \infty\) if the limit superior is zero.
    \begin{itemize}
        \item For all \(0 < r < R\), the series converges absolutely
              and uniformly to a function \(f\left( z \right)\) on the
              disk \(\abs*{z - z_0} < r\).
        \item For all \(z\) outside this disk, \(\abs*{z - z_0} > R\),
              the series diverges.
    \end{itemize}
    The convergence of the series is unknown on the boundary of the disk
    \(\abs*{z - z_0} = R\). Here \(R\) is known as the \textbf{radius of
        convergence} and can take any value in
    \(\R \cup \left\{ \infty \right\}\).
\end{theorem}
\begin{proof}
    We will first show that this series converges absolutely
    and uniformly for all \(0 < r < R\) inside the disk
    \(\abs*{z - z_0} < r\). As \(r < R\), we have \(1 / R < 1 / r\), so
    that
    \begin{equation*}
        \limsup_{n \to \infty} \abs*{a_n}^{1/n} < \frac{1}{r}.
    \end{equation*}
    Consider some \(s \in \R\), where
    \begin{equation*}
        \limsup_{n \to \infty} \abs*{a_n}^{1/n} < s < \frac{1}{r}.
    \end{equation*}
    As a consequence of the definition of the limit superior,
    \(\abs*{a_n}^{1/n}\) will be less than \(s\) for most values of
    \(n\):
    \begin{equation*}
        \abs*{a_n}^{1/n} < s \implies \abs*{a_n} < s^n.
    \end{equation*}
    If we multiply this inequality by \(\abs*{z - z_0}^n\), we find
    \begin{equation*}
        \abs*{a_n \left( z - z_0 \right)^n} < s^n \abs*{z - z_0}^n < s^n r^n = \left( s r \right)^n.
    \end{equation*}
    As \(s < 1 / r\), we have \(s r < 1\), so that the series
    \(\sum_{n = 0}^\infty \left( s r \right)^n\) converges by the
    geometric series test. Therefore, by the Weierstrass M-test, the
    series \(\sum_{n = 0}^\infty a_n \left( z - z_0 \right)^n\)
    converges absolutely and uniformly on the disk
    \(\abs*{z - z_0} < r\).
    For values of \(z\) that lie outside the disk \(\abs*{z - z_0} >
    R\), we can use the Cauchy root test to show that the series
    diverges. Consider the limit superior of the series:
    \begin{align*}
        \limsup_{n \to \infty} \abs*{a_n \left( z - z_0 \right)^n}^{1/n} & = \limsup_{n \to \infty} \abs*{a_n}^{1/n} \abs*{z - z_0} \\
                                                                         & = \abs*{z - z_0} \limsup_{n \to \infty} \abs*{a_n}^{1/n} \\
                                                                         & = \abs*{z - z_0} \frac{1}{R}                             \\
                                                                         & > 1.
    \end{align*}
    As the limit superior is greater than one, by the Cauchy root test,
    the series diverges for all \(z\) outside the disk
    \(\abs*{z - z_0} = R\).
\end{proof}
\subsubsection{Taylor Series}
The Taylor series expansion of a real-valued function \(f\left( x
\right)\) about a point \(x_0\) is defined as
\begin{equation*}
    f\left( x \right) = \sum_{n = 0}^\infty a_n \left( x - x_0 \right)^n,
\end{equation*}
where through repeated differentiation, we can show
\begin{equation*}
    a_n = \frac{f^{\left( n \right)}\left( x_0 \right)}{n!}.
\end{equation*}
Here, \(f^{\left( n \right)}\left( x_0 \right)\) is the \(n\)th
derivative of \(f\left( x \right)\) evaluated at \(x_0\).
Let us propose that we can extend this definition to a complex-valued
function \(f\left( z \right)\), such that we can define the
Taylor series about a point \(z_0\) as a complex power series:
\begin{equation*}
    f\left( z \right) = \sum_{n = 0}^\infty a_n \left( z - z_0 \right)^n,
\end{equation*}
with a radius of convergence \(R\) by the Cauchy-Hadamard theorem.
\begin{theorem}[Analytic functions have a Taylor series expansion]
    Let \(f\left( z \right)\) be an analytic function on and within a
    simple closed curve \(\mathcal{C}\), then \(f\left( z \right)\) has a
    Taylor series expansion about any point \(z_0\) within \(\mathcal{C}\):
    \begin{equation*}
        f\left( z \right) = \sum_{n = 0}^\infty a_n \left( z - z_0 \right)^n,
    \end{equation*}
    with radius of convergence \(R\), which is the largest disk from
    \(z_0\) that is contained within \(\mathcal{C}\). The coefficients
    of this series are given by:
    \begin{equation*}
        a_n = \frac{f^{\left( n \right)}\left( z_0 \right)}{n!} = \frac{1}{2 \pi i} \oint_{\mathcal{C}} \frac{f\left( w \right)}{\left( w - z_0 \right)^{n + 1}} \odif{w}.
    \end{equation*}
\end{theorem}
\begin{proof}
    Consider Cauchy's integral formula for derivatives, which states
    that
    \begin{equation*}
        f\left( z \right) = \frac{1}{2 \pi i} \oint_{\mathcal{C}} \frac{f\left( w \right)}{w - z} \odif{w}.
    \end{equation*}
    We will express the denominator term using a geometric series:
    \begin{align*}
        \frac{1}{w - z} & = \frac{1}{w - z_0 + z_0 - z}                                                                                               \\
                        & = \frac{1}{w - z_0} \cdot \frac{1}{1 - \frac{z - z_0}{w - z_0}}                                                             \\
                        & = \frac{1}{w - z_0} \sum_{n = 0}^\infty \left( \frac{z - z_0}{w - z_0} \right)^n, \qquad \abs*{\frac{z - z_0}{w - z_0}} < 1 \\
                        & = \sum_{n = 0}^\infty \frac{1}{\left( w - z_0 \right)^{n + 1}} \left( z - z_0 \right)^n.
    \end{align*}
    Therefore,
    \begin{align*}
        f\left( z \right) & = \frac{1}{2 \pi i} \oint_{\mathcal{C}} f\left( w \right) \left[ \sum_{n = 0}^\infty \frac{1}{\left( w - z_0 \right)^{n + 1}} \left( z - z_0 \right)^n \right] \odif{w} \\
                          & = \sum_{n = 0}^\infty \left[ \frac{1}{2 \pi i} \oint_{\mathcal{C}} \frac{f\left( w \right)}{\left( w - z_0 \right)^{n + 1}} \odif{w} \right] \left( z - z_0 \right)^n   \\
                          & = \sum_{n = 0}^\infty \frac{f^{\left( n \right)}\left( z_0 \right)}{n!} \left( z - z_0 \right)^n
    \end{align*}
    within the disk \(\abs*{z - z_0} < R\), where \(R\) is the radius of
    convergence.
\end{proof}
\begin{theorem}[A function with a Taylor series expansion is analytic]
    Let \(f\left( z \right)\) be a function with a Taylor series
    expansion about a point \(z_0\) with radius of convergence \(R\):
    \begin{equation*}
        f\left( z \right) = \sum_{n = 0}^\infty a_n \left( z - z_0 \right)^n,
    \end{equation*}
    Then, \(f\left( z \right)\) is analytic on and within the disk
    \(\abs*{z - z_0} < R\), and its derivatives are given by termwise
    differentiation:
    \begin{equation*}
        f^{\left( n \right)}\left( z \right) = \sum_{k = 0}^\infty a_k \odv*[order=n]{\left[ \left( z - z_0 \right)^k \right]}{z} = \sum_{k = n}^\infty a_k \frac{k!}{\left( k - n \right)!} \left( z - z_0 \right)^{k - n}.
    \end{equation*}
\end{theorem}
\begin{proof}
    As the Taylor series is a sum of analytic functions, it is also
    analytic on the disk \(\abs*{z - z_0} < R\). Proving the form of the
    derivative is trivial.
\end{proof}
\subsubsection{Singularities}
There are three types of \textbf{isolated singularities} that can occur
in complex functions. Suppose \(f\left( z \right)\) is analytic on a
punctured neighbourhood \(0 < \abs{z - z_0} < R\), but not at \(z_0\).
\begin{itemize}
    \item \textbf{Removable singularities}: \(f\left( z \right)\) has a
          removable singularity at \(z_0\) if the limit
          \begin{equation*}
              L = \lim_{z \to z_0} f\left( z \right)
          \end{equation*}
          exists and is finite. In this case, we can define a new
          function \(g\left( z \right)\)
          \begin{equation*}
              g\left( z \right) =
              \begin{cases}
                  f\left( z \right), & z \neq z_0 \\
                  L,                 & z = z_0
              \end{cases}
          \end{equation*}
          that is analytic at \(z_0\).
    \item \textbf{Poles}: \(f\left( z \right)\) has a pole of
          order \(m \in \N\) at \(z_0\) if there exists a function
          \(g\left( z \right)\) that is analytic at \(z_0\) such that
          \begin{equation*}
              f\left( z \right) = \frac{g\left( z \right)}{\left( z - z_0 \right)^m},
          \end{equation*}
    \item \textbf{Essential singularities}: \(f\left( z \right)\) has an
          essential singularity at \(z_0\) if the singularity is neither
          removable nor a pole.
\end{itemize}
\subsubsection{Laurent Series}
While Taylor series represent analytic functions within a disk centred
at a point \(z_0\), they fail to converge if the function is not
analytic at \(z_0\). To describe the behaviour of functions near
isolated singularities, we will generalise the Taylor series by allowing
negative powers of \(\left( z - z_0 \right)\).
The Laurent series expansion of a function \(f\left( z \right)\) about a
point \(z_0\) is defined as
\begin{equation*}
    f\left( z \right) = \sum_{n = -\infty}^\infty a_n \left( z - z_0 \right)^n,
\end{equation*}
where the coefficients \(a_n\) are given by the contour integral:
\begin{equation*}
    a_n = \frac{1}{2 \pi i} \oint_{\mathcal{C}} \frac{f\left( w \right)}{\left( w - z_0 \right)^{n + 1}} \odif{w}.
\end{equation*}
This series converges in an annular region defined by the radii of
convergence \(r\) and \(R\) such that
\begin{equation*}
    r < \abs*{z - z_0} < R.
\end{equation*}

\end{document}
